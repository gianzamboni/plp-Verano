\part{Apéndices}


\section{Programación funcional en Haskell}
\paragraph{Tipos elementales}
\begin{centrado}
	\begin{minted}{haskell}
1               -- Int          Enteros
'a'             -- Char         Caracteres
1.2             -- Float        Números de punto flotante
True            -- Bool         Booleanos
[1,2,3]         -- [Int]        Listas
(1, True)       -- (Int, Bool)  Tuplas, pares
length          -- [a] -> Int   Funciones
length [1,2,3]  -- Int          Expresiones
\x -> x         -- a -> a       Funciones anónimas
	\end{minted}
\end{centrado}

\paragraph{Guardas}
\begin{centrado}
	\begin{minted}{haskell}
signo n | n >= 0    = True
		| otherwise = False
	\end{minted}
\end{centrado}

\paragraph{Pattern Matching}
\begin{centrado}
	\begin{minted}{haskell}
longitud [] = 0
longitud (x:xs) = 1 + (longitud xs)
	\end{minted}
\end{centrado}

\paragraph{Polimorfismo paramétrico}
\begin{centrado}
	\begin{minted}{haskell}
todosIguales :: Eq a => [a] -> Bool
todosIguales [] = True
todosIguales [x] = True
todosIguales (x:y:xs) = x == y && todosIguales(y:xs)
	\end{minted}
\end{centrado}

\paragraph{Clases de tipo}
\begin{centrado}
	\begin{minted}{haskell}
Eq a    -- Tipos con comparación de igualdad
Num a   -- Tipos que se comportan como los números
Ord a   -- Tipos orden
Show a  -- Tipos que pueden ser representados como strings
	\end{minted}
\end{centrado}

\paragraph{Definición de listas}
\begin{centrado}
	\begin{minted}[breaklines]{haskell}
[1,2,3,4,5]                 -- Por extensión
[1 .. 4]                    -- Secuencias aritméticas
[ x | x <- [1..], esPar x ] -- Por compresión
		
cuando las usamos. Ejemplo de lista infinita:
		
infinitosUnos :: [Int]
infinitosUnos = 1 : infinitosUnos
		
puntosDelCuadrante :: [(Int, Int)]
puntosDelCuadrante = [ (x, s-x) | s <- [0..], x <-[0..s] ]
	\end{minted}
\end{centrado}

\paragraph{Funciones de alto orden}
\begin{centrado}
	\begin{minted}[breaklines]{haskell}
mejorSegun :: (a -> a -> Bool) -> [a] -> a
mejorSegun _ [x] = x
mejorSegun f (x : xs) | f x (mejorSegun f xs) = x
		| otherwise = mejorSegun f xs
	\end{minted}
\end{centrado}

\subsection{Otros tipos útiles}
\paragraph{Formula}
\begin{centrado}
	\begin{minted}[breaklines]{haskell}
data Formula = Proposicion String | No Formula 
		| Y Formula Formula
		| O Formula Formula
		| Imp Formula Formula
		
foldFormula :: (String -> a) -> (Formula -> a) -> 
(Formula -> Formula -> a) -> (Formula -> Formula -> a) 
-> (Formula -> Formula -> a) -> Formula -> a
foldFormula fp fn fy fo fImp form = case form of :
		Proposicion s -> fp s
		No sf -> fn (rec sf)
		Y sf1 sf2 -> fy (rec sf1) (rec sf2)
		O sf1 sf2 -> fo (rec sf1) (rec sf2)
		Impl sf1 sf2 -> fImpl (rec sf1) (rec sf2)
		where rec = foldForm fp fn fy fo fImp
	\end{minted}
\end{centrado}

\paragraph{Rosetree}
\begin{centrado}
	\begin{minted}[breaklines]{haskell}
data Rosetree = Rose a [Rosetree]
-- Hay varias formas de definir el fold para esta estructura
foldRose :: (a -> [b] -> b) -> Rosetree a -> b
foldRose f ( Rose x l ) = f x ( map ( foldRose f ) l )
		
foldRose2 :: ( a -> c -> b) -> ( b -> c -> c ) -> c 
-> Rosetree a -> b
foldRose2 g f z (Rose x l) = 
g x ( foldr f z ( map ( foldRose g f z ) l ) )
		
	\end{minted}
\end{centrado}


\newpage
\section{Extensiones del lenguaje \texorpdfstring{\(\lambda^b\)}{lambda b}}



\subsection{Registros \texorpdfstring{\(\lambda^{...r}\)}{lambda ...r}}

\paragraph{Tipos}
\[\sigma, \tau ~::=~...~|~\{l_i : \sigma_i ~^{i\in 1..n}\}\]

El tipo \(\{l_i : \sigma_i^{i\in 1..n}\}\) representan las estructuras con \(n\) atributos tipados, por ejemplo: \(\{nombre : String,edad:Nat\}\)
\paragraph{Términos}
\[ M~::=~ \dots~|~\{l_i = M_i ~^{i\in 1..n}\}~|~M.l \]

Los términos significan:
\begin{itemize}
	\item El registro \(\{l_i = M_i ~^{i\in 1..n}\}\) evalua \(\{l_i = V_i ~^{i\in 1..n}\}\)  donde \(V_i\) es el s al que evalúa \(M_i\) para \(i\in 1..n\).
	\item \(M.l\): Proyecta el valor de la etiqueta \(l\) del registro \(M\)
\end{itemize}

\paragraph{Axiomas y reglas de tipado}
\begin{equation*}
	\frac{\judgeType{\Gamma}{M_i}{\sigma_i} \text{ para cada } i \in 1..n}{\judgeType{\Gamma}{\{l_i = M_i ~^{i\in 1..n}\}}{\{l_i : \sigma_i ~^{i\in 1..n}\}}}(\text{T-RCD})
\end{equation*}
\vspace*{5mm}
\begin{equation*}
	\frac{\judgeType{\Gamma}{\{l_i = M_i ~^{i\in 1..n}\}}{\{l_i : \sigma_i ~^{i\in 1..n}\}}\hspace*{5mm} j \in 1..n}
	{\judgeType{\Gamma}{M.l_j}{\sigma_j}}(\text{T-Proj})
\end{equation*}

\paragraph{Axiomas y reglas de subtipado}
%\begin{equation*}
%	\frac{}{\{l_i : \sigma_i~|~i\in 1..n+k\} <: \{l_i : \sigma_i~|~i\in 1..n\}}(\text{S-RcdWidth})
%\end{equation*}
%
%\vspace*{5mm}
%\begin{equation*}
%	\frac{\sigma_i <: \tau_i\hspace*{5mm} i\in I = \{1..n\}}{\{l_i : \sigma_i\}_{i\in I} <: \{l_i : \tau_i\}_{i\in I}}(\text{S-RcdDepth})
%\end{equation*}

\begin{equation*}
	\frac{\{l_i| i\in 1..n\}\subseteq\{k_j|j\in 1..m\} \hspace*{5mm} k_j = l_i\Rightarrow \sigma_j <: \tau_i}{\{k_j:\sigma_j|j\in 1..m\} <: \{l_i:\sigma_i| i\in 1..n\}}(\text{S-Rcd})
\end{equation*}

\vspace*{5mm}
Esta regla nos dice que un registro \(N\) es subtipo de otro registro \(M\), si el conjunto de etiquetas de \(M\) está contenido en el conjunto de etiquetas de \(N\) y, además, si los tipos de cada una de esas etiquetas, en \(M\), es más general que en \(N\).

Una de las consecuancias de esta regla es que  \(\sigma <\{\}\) para todo tipo registro \(\sigma\). Esto es porque \(\{\}\) no tiene etiquetas, osea que su conjunto de etiquetas es el conjunto \(\emptyset\) que está contenido en todos los conjuntos.
\paragraph{Valores}
\[V~::=~\dots~|~\{l_i = V_i ~^{i\in 1..n}\}\]


\paragraph{Axiomas y reglas de evaluación}

\begin{equation*}
	\frac{j\in 1..n}{\{l_i = \lambdaValue{V_i} ~^{i\in 1..n}\}.l_j \to \lambdaValue{V_j}}(\text{E-ProjRcd})
\end{equation*}
\vspace*{5mm}
\begin{equation*}
	\frac{M \to M'}{M.l \to M'.l}(\text{E-Proj})
\end{equation*}

\vspace*{5mm}
\begin{equation*}
	\frac{M_j\to M_j'}{\{l_i = \lambdaValue{V_i}~^{i\in 1..j-1}, l_j = M_j, l_i = M_i ~^{i\in j+1..n}\} \to \{l_i = \lambdaValue{V_i}~^{i\in 1..j-1}, l_j = M'_j, l_i = M_i ~^{i\in j+1..n}\}}(\text{E-RCD})
\end{equation*}
\vspace*{5mm}
\subsection{Declaraciones Locales (\texorpdfstring{\(\lambda^{...let}\)}{lambda ...let})}\label{extension_lambda:let}

Con esta extensión, agregamos al lenguaje el término \(\lambdaLet{x}{\sigma}{M}{N}\), que evalúa \(M\) a un valor, liga \(x\) a \(V\) y, luego, evalúa \(N\). Este término solo mejora la legibilidad de los programas que ya podemos definir con el lenguaje hasta ahora definido.

\paragraph{Términos}
\[ M~::=~ \dots~|~\lambdaLet{x}{\sigma}{M}{N} \]


\paragraph{Axiomas y reglas de tipado}
\begin{equation*}
	\frac{\judgeType{\Gamma}{M}{\sigma_1}\hspace*{5mm}\judgeType{\Gamma,x:\sigma_1}{N}{\sigma_2}}{\judgeType{\Gamma}{\lambdaLet{x}{\sigma_1}{M}{N}}{\sigma_2}}(\text{T-Let})
\end{equation*}

\paragraph{Axiomas y reglas de evaluación}

\begin{equation*}
	\frac{M_1\to M_1'}{\lambdaLet{x}{\sigma}{M_1}{M_2}\to \lambdaLet{x}{\sigma}{M'_1}{M_2}}(\text{E-Let})
\end{equation*}
\vspace*{5mm}
\begin{equation*}
	\frac{}{\lambdaLet{x}{\sigma}{\lambdaValue{V_1}}{M_2}\to \replaceBy{M_2}{x}{\lambdaValue{V_1}}}(\text{E-LetV})
\end{equation*}

\subsubsection{Construcción \texorpdfstring{\textit{let}}{let} recursivo (Letrec)}
Una construcción alternativa para definir funciones recursivas es 
\[letrec~f:\sigma\to\sigma = \lambdaAbs{x}{\sigma}{M~in~N}\]

Y \(letRec\) se puede definir  en base a \(let\) y \(fix\) (definido en \ref{lambda_calculo:recursion}) de la siguiente forma:

\[\lambdaLet{f}{\sigma\to\sigma}{(\lambdaFix{\lambdaAbs{f}{\sigma\to\sigma}{\lambdaAbs{x}{\sigma}{M}}})}{N}\]

\subsection{Tuplas}

\paragraph{Tipos}
\[\sigma,\tau~::= \dots~|~\sigma\times\tau\]

\paragraph{Términos}
\[M,~N~::=~\dots~|~<M,N>~|~\pi_1(M)~|~\pi_2(M)\]
\paragraph{Axiomas y reglas de tipado}
\begin{equation*}
	\frac{\judgeType{\Gamma}{M}{\sigma}\hspace*{5mm}\judgeType{\Gamma}{N}{\tau}}{\judgeType{\Gamma}{<M,N>}{\sigma\times\tau}}(\text{T-Tupla})
\end{equation*}
\vspace*{5mm}
\begin{equation*}
	\frac{\judgeType{\Gamma}{M}{\sigma\times\tau}}{\judgeType{\Gamma}{\pi_1(M)}{\sigma}}(\text{T-}\pi_1)\hspace*{1cm}\frac{\judgeType{\Gamma}{M}{\sigma\times\tau}}{\judgeType{\Gamma}{\pi_2(M)}{\tau}}(\text{T-}\pi_2)
\end{equation*}

\paragraph{Valores}
\[V~::=~\dots~|~<V,V>\]

\paragraph{Axiomas y reglas de evaluación}
\[
  \frac{M\to M'}{<M,N>\to<M',N>}(\text{E-Tuplas})\hspace*{1cm}\frac{N\to N'}{<\lambdaValue{V},N>\to<\lambdaValue{V},N'>}(\text{E-Tuplas1})
\]
\vspace*{5mm}
\begin{equation*}
	\frac{M\to M'}{\pi_1(M)\to\pi_1(M')}(\text{E-}\pi_1)\hspace*{1cm}\frac{}{\pi_1(<\lambdaValue{V_1}, \lambdaValue{V_2}>)\to\lambdaValue{V_1}}(\text{E-}\pi'_1)
\end{equation*}
\vspace*{5mm}
\begin{equation*}
	\frac{M\to M'}{\pi_2(M)\to\pi_2(M')}(\text{E-}\pi_2)\hspace*{1cm}\frac{}{\pi_2(<\lambdaValue{V_1}, \lambdaValue{V_2}>)\to\lambdaValue{V_2}}(\text{E-}\pi'_2)
\end{equation*}

\subsection{Listas}

\paragraph{Tipos}
\[\sigma::= \dots~|~[\sigma]\]

\paragraph{Términos}
\[M,~N~,O~::=~\dots~|~[]_{\sigma}~|~M::N~|~\texttt{case } M \texttt{ of } {[] \leadsto N | h::t \leadsto O}~|~\text{foldr } M \text{ base} \leadsto N; \text{rec}(h,r)\leadsto O\]

Los términos son:
\begin{itemize}
	\item \([]_{\sigma}\) es la lista vacía de tipo \(\sigma\)
\end{itemize}
\paragraph{Axiomas y reglas de tipado}
\begin{multicols}{2}
	\[\frac{}{\judgeType{\Gamma}{\List{\sigma}}{[\sigma]}}(\text{T-Vacio})\]
	
	\vspace*{5mm}
	\[\frac{\judgeType{\Gamma}{M}{\sigma}\hspace*{5mm}\judgeType{\Gamma}{N}{[\sigma]}}{\judgeType{\Gamma}{M::N}{[\sigma]}}(\text{T-}::)\]
	
	\vspace*{5mm}
	\[\frac{\judgeType{\Gamma}{M}{[\sigma]}\hspace*{5mm}\judgeType{\Gamma}{N}{\tau}~\hspace*{5mm}\judgeType{\Gamma,h:\sigma,t:[\sigma]}{O}{\tau}}{\judgeType{\Gamma}{\lambdaListCase{M}{N}{O}}{\tau}}(T-Case)\]
	
	\vspace*{5mm}
	\[\frac{\judgeType{\Gamma}{M}{[\sigma]}\hspace*{5mm}\judgeType{\Gamma}{N}{\tau}~\hspace*{5mm}\judgeType{\Gamma,h:\sigma,r:\tau}{O}{\tau}}{\judgeType{\Gamma}{\lambdaListFold{M}{N}{O}}{\tau}}(T-Fold)\]
	
\end{multicols}

\paragraph{Valores}
\(V~::= ...~|~\List{\sigma}~|~V::V\)

\paragraph{Axiomas y reglas de evaluación}
\begin{multicols}{2}
	\[\frac{M_1\to M_1'}{M_1 :: M_2 \to M'_1::M_2}(\text{E-}::\text{1})\]
	
	\vspace*{5mm}
	\[\frac{M_2\to M_2'}{V :: M_2 \to V::M'_2}(\text{E-}::\text{2})\]
	
	\vspace*{5mm}
	\[\frac{}{\lambdaListCase{\List{\sigma}}{N}{O}\to N}(\text{E-Case}[~])\]
	
\end{multicols}
	\vspace*{5mm}
\[\frac{}{\lambdaListCase{V_1::V_2}{N}{O}\to\multiReplaceBy{O}{h\leftarrow V_1,~t\leftarrow V_2}}(\text{E-Case}::)\]

\vspace*{5mm}
\[\frac{M\to M'}{\lambdaListCase{M}{N}{O}\to\lambdaListCase{M'}{N}{O}}(\text{E-Case}\text{1})\]

\vspace*{5mm}
\[\frac{}{\lambdaListFold{\List{\sigma}}{N}{O}\to N}(\text{E-Fold}[~])\]

\vspace*{5mm}
\[\frac{}{\lambdaListFold{V_1::V_2}{N}{O}\to \multiReplaceBy{O}{h\leftarrow V_1,~r\leftarrow(\lambdaListFold{V_2}{N}{O})}}(\text{E-Fold}::)\]

\vspace*{5mm}
\[\frac{M\to M'}{\lambdaListFold{M}{N}{O}\to\lambdaListFold{M'}{N}{O}}(\text{E-Fold}\text{1})\]


\subsection{Árboles binarios}

\paragraph{Tipos}
\[\sigma,\tau~::= \dots~|~AB_\sigma\]

\paragraph{Términos}
\[M,~N~::=~\dots~|~\text{Nil}_\sigma~|~\text{Bin}(M, N, O)~|~\text{raiz}(M)~|~\text{der}(M)~|~\text{izq}(M)~|~\text{esNil}(M)\]
\paragraph{Axiomas y reglas de tipado}
\begin{equation*}
	\begin{gathered}
		\frac{}{\judgeType{\Gamma}{\text{Nil}_\sigma}{AB_\sigma}}(\text{T-Nil})\hspace*{1cm}
		\frac{\judgeType{\Gamma}{M}{AB_\sigma}\hspace*{5mm}\judgeType{\Gamma}{N}{\sigma}\hspace*{5mm}\judgeType{\Gamma}{O}{AB_\sigma}}{\judgeType{\Gamma}{\text{Bin}(M, N, O)}{AB_\sigma}}(\text{T-Bin}) \\
		\vspace*{5mm}\\
		\frac{\judgeType{\Gamma}{M}{AB_\sigma}}{\judgeType{\Gamma}{\text{raiz}(M)}{\sigma}}(\text{T-raiz})\hspace*{1cm}
		\frac{\judgeType{\Gamma}{M}{AB_\sigma}}{\judgeType{\Gamma}{\text{der}(M)}{AB_\sigma}}(\text{T-der})
		\vspace*{5mm} \\
		\frac{\judgeType{\Gamma}{M}{AB_\sigma}}{\judgeType{\Gamma}{\text{izq}(M)}{AB_\sigma}}(\text{T-izq})
		\hspace*{1cm}
		\frac{\judgeType{\Gamma}{M}{AB_\sigma}}{\judgeType{\Gamma}{\text{isNil}(M)}{Bool}}(\text{T-isNil})
	\end{gathered}
\end{equation*}

\paragraph{Valores}
\[V~::=~\dots~|~\text{Nil}~|~\text{Bin}(V,V,V)\]

\paragraph{Axiomas y reglas de evaluación}
\begin{equation*}
	\frac{M\to M'}{\text{Bin}(M,N,O)\to \text{Bin}(M',N,O)}(\text{E-Bin1})\hspace*{1cm}\frac{N\to N'}{\text{Bin}(V,N,O)\to \text{Bin}(V,N',O)}(\text{E-Bin2})
\end{equation*}
\vspace*{5mm}
\begin{equation*}
	\frac{O\to O'}{\text{Bin}(V_1,V_2,O)\to \text{Bin}(V_1,V_2,O')}(\text{E-Bin3})
\end{equation*}
\vspace*{5mm}
\begin{equation*}
	\frac{M\to M'}{\text{raiz}(M)\to\text{raiz}(M')}(\text{E-Raiz1})\hspace*{1cm}\frac{}{\text{raiz}(\text{Bin}(V_1,V_2,V_3))\to V_2}(\text{E-Bin3})
\end{equation*}
\vspace*{5mm}
\begin{equation*}
	\frac{M\to M'}{\text{der}(M)\to\text{der}(M')}(\text{E-Der1})\hspace*{1cm}\frac{}{\text{der}(\text{Bin}(V_1,V_2,V_3))\to V_3}(\text{E-Der2})
\end{equation*}
\vspace*{5mm}
\begin{equation*}
	\frac{M\to M'}{\text{izq}(M)\to\text{izq}(M')}(\text{E-Izq1})\hspace*{1cm}\frac{}{\text{izq}(\text{Bin}(V_1,V_2,V_3))\to V_1}(\text{E-Izq2})
\end{equation*}
\hspace*{5mm}
\begin{equation*}
	\frac{}{\text{isNil}(M)\to\text{izq}(M')}(\text{E-isNil1})\hspace*{1cm}\frac{}{\text{isNil}(\text{Bin}(V_1,V_2,V_3))\to false}(\text{E-isNilBin})
\end{equation*}
\hspace*{5mm}
\begin{equation*}
	\frac{}{\text{isNil}(\text{Bin}(V_1,V_2,V_3))\to true}(\text{E-isNilNil})
\end{equation*}

\newpage
\section{Javascript}
\paragraph{Declaración de variables}
\begin{centrado}
	\begin{minted}{javascript}
	// Declaración de variables
	let miVar = 1;
	var suVar = 2;
	
	// Declaración de constante, no pueden ser modificadas.
	const miConstante = 3; 
	\end{minted}
\end{centrado}

Y es \textbf{case-sensitive}, es decir \javascript{unavariable} y \javascript{unaVariable} no son las mismas variables.

\paragraph{Tipos}

\begin{itemize}
	\item \javascript{number}: Los números, no hay distinción entre enteros y punto flotantes. Contiene a las constantes \javascript{-Infinity},\javascript{+Infinity}, \javascript{NaN}.
	\item \javascript{boolean}: Los literales \javascript{true} y \javascript{false} con las operaciones \javascript{\&\&}, \javascript{!} y \javascript{||}.
	\item \javascript{string}: Secuencias de cero o más carácteres entre comillas simples o dobles.
	\item \javascript{null}: Un único valor \javascript{null} (nada, valor desconocido).
	\item \javascript{undefined}: Un único valor \javascript{undefined} (el valor no está definido).
\end{itemize}

Podemos usar \javascript{typeof} para saber el nombre del tipo de la expresión.

\begin{itemize}
\item \javascript{Arrays:} \javascript{[], [1,2,true], new Array()}

\javascript{-[-] , push(-) , pop() , shift() , unshift(_)}
\end{itemize}

\paragraph{Tipado}
El tipado se hace de manera \textbf{dinámica} (en tiempo de ejecución) y \textbf{débil} (se pueden comparar cosas que no son del mismo tipo porque hay conversión automática).

Por ejemplo:
\begin{centrado}
\begin{minted}[breaklines]{javascript}
let a = 1  // a = 1
a += '1'
// a = '11' (El entero, se convierte automaticamente en un string)
\end{minted}

\end{centrado}

\begin{multicols}{2}
\begin{minted}{javascript}
1 == '1' // true
1 === '1' // false
0 == false // true
0 === false // false
1 == true // true
\end{minted}
\vfill\null
\columnbreak
\begin{minted}{javascript}
false == '' // true
false === 'false' // false
null == undefined // true
null === undefined // false
\end{minted}
\end{multicols}

\newpage
\paragraph{Flujos de control}
\begin{centrado}
\begin{minted}{javascript}
if (cond) { ... } else { ... }

while (cond) {
    //cuerpo
}

do {
    //cuerpo
} while (cond);
\end{minted}

\end{centrado}

\paragraph{Definición de funciones}
\begin{centrado}
\begin{minted}{javascript}
function nombre(arg1, ..., argn){
    //cuerpo
}

let nombre = function(arg1, ..., argn){
    //cuerpo
}

let nombre = (arg1, ..., argn) => {
    //cuerpo
}
\end{minted}

\end{centrado}

\paragraph{Objetos}
\begin{centrado}
\begin{minted}[breaklines]{javascript}
let o = {
    a : 1,
    b : function(n) {
        return this.a + n // this hace referencia a o
    }
}

o.a // 1 Proyeccion del atributo a

o['a'] // 1 (Proyeccion del atributo a)

o.b(1) // 2  (Proyeccion y ejecucion del metodo b)


o.b = function() { return this.a } // Redefinimos o.b

o.c = true // Agrega el atributo c a o.
o['c'] = false // Redefinimos el atributo c de o.
delete o.a // Elimina el atributo a de o.


\end{minted}
\end{centrado}

\begin{centrado}
\begin{minted}[breaklines]{javascript}
'c' in o // true (Checkea si c es una propiedad de o)
'd' in o // false

// Iteracion sobre todas las propiedades de o.
for(let p in o){
    ...
}
\end{minted}
\end{centrado}
\begin{centrado}
\begin{minted}[breaklines]{javascript}

let p = o // Creamos una referencia a o.

let f = o.b 
// Extraemos el metodo b de f. Dejamos las variable this desligada.

let o2 = { i: f, a: true} 
// Crea el objeto o2, con la función f en su atributo i.

o2.f() // true

let o3 = Object.assign({}, o, o2) 
// Copia las propiedade de o y o2 en o3. Hace un shallow copy, es decir si una atributo de o2 u o3 es una referencia, entonces en o, ese atributo va a ser una referencia al mismo objeto que en o2
\end{minted}
\end{centrado}

\paragraph{Herencia}
Todos los objetos tiene una propiedad privada llamada \javascript{[[Prototype]]} cuyo valor es \javascript{null} u otro objeto que es su \textbf{prototipo}. Esta propiedad induce una cadena de prototipado sin ciclos que finaliza con \javascript{null}.

Cuando intentamos acceder a un método inexistente de un objeto, el mismo se busca en la cadena de prototipado del mismo hasta encontrar el primer objeto de la cadena que lo define. Si llegamos a \javascript{null} y el método no fue encontrado, entonces hay un error.

\begin{centrado}
\begin{minted}[breaklines]{javascript}
Object.setPrototypeOf(o, prot) // Hace que el prototipo de o sea prot.
Object.getPrototypeOf(o) // Devuelve el objeto que es protipo de o.

o.__proto__ // Otra forma de conseguir el prototipo de o.
o.__proto__ = o1 // Otra forma de setear el prototipo de o
\end{minted}
\end{centrado}

Cuando ejecutamos un método de la cadena de prototipado, el método evaluado liga \javascript{this} al objeto llamó al método, es decir, si tenemos:

\begin{centrado}
\begin{minted}[breaklines]{javascript}
let o1 = { a: 1}
let o2 = { b: function(){
         return this.a
    }
}
\end{minted}
\end{centrado}
\begin{centrado}
\begin{minted}[breaklines]{javascript}
o1.__proto__= o2

o1.b();
// Busca b en o1 y no lo encuentra, el proximo objeto en la cadena es o2, encuentra b y liga this = o1, luego ejecuta el método.
\end{minted}
\end{centrado}

Para crear una copia de un objeto y asignarle a esa copia el objeto original como prototipo usamos \javascript{Object.create}.

Además javascript provee el prototipo \javascript{Object.prototype} que es prototipo de todos los objetos y provee metodos básicos como \javascript{hasOwnProperty()} que indica si el objeto contiene una propiedad no heredada y \javascript{toString()} que devuelve un string que representa al receptor.

\paragraph{Cadenas de prototipado}

\begin{centrado}
\begin{minted}[breaklines]{javascript}
let o1 = { ... }
// o1 ---> Object.prototype ---> null

let o2 = Object.create(o1)
// o2 ---> o1 ---> Object.prototype ---> null

let o = Object.create(null)
// o ---> null
\end{minted}
\end{centrado}

\paragraph{Constructores} Son funciones que generan objetos. Cuando declaramos un constructor \javascript{C}, javascript crea un objeto llamado \javascript{C.prototype} que se asigna como prototipo de todos los objetos creados con dicha función. Por ejemplo:

\begin{centrado}
\begin{minted}[breaklines]{javascript}
    // Constructor Punto, tiene la siguiente cadena de prototipado:
    // Punto --> Function.prototype --> Object.prototype --> null
    function Punto(x,y){
        this.x = x;
        this.y = y;
        this.mvx = function(d){
            this.x += d;
        }
    }
    
    // Objeto creado con el constructo punto
    o = new Punto(1,2)
    // o = Object{ x: 1, x:2, mvx: mvx()} y su cadena de prototipado es:
    // o --> Punto.prototype --> Object.prototype --> null
\end{minted}
\end{centrado}
