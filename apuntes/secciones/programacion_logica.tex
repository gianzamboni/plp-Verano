En este paradigma, las programas son un conjunto \textbf{hechos} y \textbf{reglas de inferencia} que sirven para inferir si un \textbf{objetivo} o \textbf{goal} es consecuencia de ellos. Es decir que, cuando escribimos un programa, \textbf{declaramos} expresiones que sabemos que son verdad (hechos y reglas) que nos permitirán probar, a través del uso de la lógica, que una expresión (objetivo) es verdad.

En prolog, esta demostración, se hace a través de un motor de inferencia que se basa en el \textbf{método de resolución} para realizar la demostración. Vamos a ver métodos no deterministicos de resolución para expresiones de la lógica proposicional y de primer orden. Luego, agregaremos a estos métodos reglas que nos permitirán convertirlos en algoritmos determinísticos y las formas y condiciones que debe cumplir un programa para que sea compatible con estas reglas.
