\documentclass[10pt,a4paper]{article}

\usepackage[pdftex,
pdfauthor={Gianfranco Zamboni},
pdftitle={Resumen: Paradigmas de Lenguajes de Programación},
pdfsubject={},
pdfkeywords={Resumen , Computacion, FCEyN, UBA, Paradigmas de Lenguajes de Programación, Imperativo, Funcional, Cálculo Lambda, Programación Orientada a Objetos, Objetos, Programación Lógica},
pdfproducer={Latex with hyperref},
pdfcreator={pdflatex}]{hyperref}

\usepackage{amsmath}
\usepackage{ amssymb }

\usepackage[spanish]{babel}
\usepackage[utf8]{inputenc} % para poder usar tildes en archivos UTF-8
\usepackage{minted}
\usepackage{a4wide} % márgenes un poco más anchos que lo usual
\usepackage[titletoc,toc,page]{appendix}

\newenvironment{centrado}
    {
     \begin{center}
     \begin{minipage}{0.8\textwidth}
 }    
    {
     \end{minipage}
     \end{center}
    }

\newcommand{\rel}{\ensuremath{\mathcal{R}}}

\renewcommand{\appendixtocname}{Ap\'endices}
\renewcommand{\appendixpagename}{Ap\'endices}

\input{page.layout}

\begin{document}
\title{Resumen: Paradigmas de Lenguajes de Programación}

\date{\today}

\author{Zamboni, Gianfranco}

\maketitle
\tableofcontents

\newpage
\setcounter{page}{1}

Estos son los apuntes de la clases de PLP que se dio en Verano 2018. Prácticamente es una combinación de las diapositivas con lo que anoté de la teórica y las prácticas pero puede tener errores. En caso de ser así estaría bueno que me avisen así los corrigo. 

El tema de objetos en esta cursada se dió distinto a como se venía dando en cursadas anteriores. Vimos prototipado en vez de clasificación y además definimos el cálculo $\varsigma$ análogo al cálculo $\lambda$ en funcional. Los profesores no estaban seguros si estas modificaciones se iban a mantener o no durante las próximas cursadas.

\section{Introducción}

\paragraph{Paradigma} Marco filosófico y teórico de una escuela científica o disciplina en la que se formulan teorías, leyes y generalizaciones y se llevan a cabo experimentos que les dan sustento.

\paragraph{Lenguaje de programación} Es un lenguaje usado para comunicar instrucciones a una computadora. Éstas describen los cómputos que debe llevar a cabo.

Un lenguaje de programación es computacionalmente completo si puede expresar todas las funciones computables.

\paragraph{Paradigma de lenguaje de programación} Marco filosófico y teórico en el que se formulan soluciones a problemas de naturaleza algorítmica. Lo entendemos como un estilo de programación en el que se escriben soluciones a problemas en términos de algoritmos.

Su ingrediente básico es el modelo de cómputo, que es la visión que tiene el usuario de cómo se ejecutan sus programas.

\subsection{Aspectos del lenguaje}

\paragraph{Sintaxis} Descripción del conjunto de secuencias de símbolos considerados como programas válidos. Nos indica cuales son los símbolos del lenguaje y como combinarlos para que se les pueda dar una semántica.

\subsubsection{Semántica}

Descripción del significado de instrucciones y expresiones. Permite asignarle un significado a aquellas expresiones que formen parte de algún lenguaje, sea informal (e.g. Castellano) o formal (basado en técnicas matemáticas).

Dependiendo el tipo de semántica que se esté utilizando, podremos interpretar un programa de distintas maneras:  Si $A$ es el dominio del problema y $B$ la imagen, entonces:

\paragraph{Semántica operacional} Se ve a un programa como un mecanismo que, dado un elemento $a\in A $, sigue una sucesión de pasos para calcular el elemento que le corresponde en $B$ a $a$.

\paragraph{Semántica axiomática} Interpreta a un programa como un conjunto de propiedades verdaderas que indican los estados que puede llegar a tomar ciertos valores.

\paragraph{Semántica denotacional} Un programa es un valor matemático (función) que relaciona cada elemento de $A$ (expresiones que lo componen) con un único elemento de $B$ (significado de las expresiones).

\subsubsection{Sistema de tipo}
Es una herramienta que nos permite analizar código para prevenir errores comunes en tiempo de ejecución (e.g. evitar sumar booleanos, usar funciones con un número incorrecto de argumentos, etc). En general, requiere anotaciones de tipo en el código fuente. 

Además sirve para que la especificación de un programa sea más clara.

Hay dos clases de análisis de tipos:
\begin{itemize}
	\item \textbf{Estático}: En tiempo de compilación.
	\item \textbf{Dinámico}: En tiempo de ejecución.
\end{itemize}

\subsection{Paradigmas}
\subsubsection{Paradigma Imperativo}

\paragraph{Estado global} Se usan variables que representan celdas de memoria en distintos momentos del tiempo. Se usan para ir almacenando resultados intermedios del problema.

\paragraph{Asignación} Es la acción que modifica las variables.

\paragraph{Control de flujo} Es la forma que tenemos de controlar el orden y la cantidad de veces que se repite un cómputo dentro del programa. En este paradigma, la repetición de cómputos se basa en la iteración.

\vspace*{5mm}

Por lo general, los lenguajes de este paradigma son eficientes ya que el modelo de ejecución usado y la arquitectura de las computadoras (a nivel procesador) son parecidos. Sin embargo, el bajo nivel de abstracción que nos proveen hacen que la implementación de un problema sea difícil de entender.

\subsubsection{Paradigma Funcional}
No tiene un estado global. Un cómputo se expresa a través de la aplicación y composición de funciones y los resultados intermedios (salida de las funciones) son pasados directamente a otras funciones como argumentos. Todas las expresiones de este paradigma son tipadas y usa la recursión para repetir cómputos.

Ofrece un alto nivel de abstracción, es declarativo, usa una matemática elegante y se puede usar razonamiento algebraico para demostrar correctitud de programas.

\subsubsection{Paradigma Lógico}
Los programas son predicados de la lógica proposicional y la computación esta expresada a través de proof search (probar que el predicado expresado es verdadero bajo ciertos axiomas). No existe un estado global y los resultados intermedios son pasados por unificación. La repetición se basa en la recursión.

Ofrece un alto nivel de abstracción, es muy declarativo y, al ser predicados, tiene fundamentos lógicos robustos pero su ejecución es muy lenta.

\subsubsection{Paradigma Orientado a Objetos}
La computación se realiza a través del intercambio de mensajes entre objetos. Tiene dos enfoques: basados en clases o basados en prototipos.

Ofrece alto nivel de abstracción y arquitecturas extensibles pero usa una matemática de programas compleja.

\newpage

\input{paradigma_funcional}
\newpage

\section{Cálculo Lambda Tipado}
El cálculo lambda es un modelo de computación turing completo basado en \textbf{funciones} introducido por \textbf{Alonzo Church}. Este modelo consiste en un conjunto de expresiones que representan abstracciones (aplicaciones de funciones) y cuyos valores pueden ser determinados aplicando ciertas reglas sintácticas de manera iterativa hasta obtener lo	que se dice su forma normal. Ésta es una expresión que, a falta de reglas, no puede ser reducida de ninguna manera. En nuestro caso, estamos estudiando cálculo lambda tipado, es decir que habrá expresiones que, a pesar de estar bien formadas, no tendrán sentido.

% TODO https://en.wikipedia.org/wiki/Hindley%E2%80%93Milner_type_system

\subsection{Expresiones de Tipos de \texorpdfstring{$\lambda^b$}{lambda b}}
El primer lenguaje lambda que usamos en la materia tiene dos \textbf{tipos} $Bool$ y $\sigma\rightarrow\theta$ que son los tipos de los valores booleanos y las funciones que van de un tipo $\sigma$ a un tipo $\theta$, respectivamente. Y lo notamos:
\begin{equation*}
	\sigma,\theta ~::=~ Bool ~|~ \sigma\rightarrow\theta
\end{equation*}

\subsubsection{Términos de \texorpdfstring{$\lambda^b$}{lambda b}}
Debemos definir los \textbf{términos} que nos permitirán escribir las expresiones válidas del tipado. Sea $\mathcal{X}$ un conjunto infinito enumerable de variables y $x\in\mathcal{X}$. Los \textbf{términos} de $\lambda^b$ están dados por:

\begin{equation*}
	\begin{split}
		M, P, Q ~ ::&= ~ true \\
		& |~ false \\
		& |~ \lambdaIf{M}{P}{Q} \\
		& |~ \lambdaApp{M}{N} \\
		& |~ \lambdaAbs{x}{\sigma}{M} \\
		& |~ x
	\end{split}
\end{equation*}

Esto significa que dados tres términos $M$, $P$ y $Q$, los términos válidos del lenguaje son:
\begin{itemize}
	\item $true$ y $false$: Representan las \textbf{constantes de verdad}.
	\item $ \lambdaIf{M}{P}{Q}$: Expresa el \textbf{condicional}.
	\item $\lambdaApp{M}{N}$: Indica la \textbf{aplicación} de la función denotada por el termino $M$ al argumento $N$.
	\item $\lambdaAbs{x}{\sigma}{M}$: \textbf{Función} (abstracción) cuyo parámetro $x$ es de tipo $\sigma$ y cuyo cuerpo es $M$
	\item $x$, una \textbf{variable de términos}.
\end{itemize}

\subsubsection{Variables ligadas y libres}
Por como definimos el lenguaje, una variable $x$ puede ocurrir de dos formas: \textbf{libre} o \textbf{ligada}. Decimos que $x$ ocurre \textbf{libre} si no se encuentra bajo el alcance de una ocurrencia de $\lambda x$. Caso contrario ocurre ligada.

Por ejemplo:
$$\lambdaAbs{x}{Bool}{\lambdaIf{true}{\underbrace{x}_{ligada}}{\underbrace{y}_{libre}}} $$

Definimos la función $FV$ (free variables) que devuelve el conjunto de variables libres de una expresión dada:

\begin{equation*}
	\begin{split}
		FV(x) &\equalDef {x} \\
		FV(true) = FV(false) &\equalDef \emptyset \\
		FV(\lambdaIf{M}{P}{Q}) &\equalDef FV(M)\cup FV(P)\cup FV(Q) \\
		FV(\lambdaApp{M}{N}) &\equalDef FV(M)\cup FV(N) \\
		FV(\lambdaAbs{x}{\sigma}{M}) &\equalDef FV(M) - \{x\}
	\end{split}
\end{equation*}

\subsubsection{Reglas de sustitución}
Una de las operaciones que podemos realizar sobre las expresiones del lenguaje es la \textbf{sustitución}. Dado un término $M$, sustituye todas las ocurrencias \textbf{libres} de una variable $x$ por un término $N$. La notamos:

$$\replaceBy{M}{x}{N}$$

Esta operación nos sirve para darle semántica a la aplicación de funciones y es sencilla de definir, sin embargo debemos tener en cuenta algunos casos especiales.

\paragraph{$\alpha$-equivalencia} Dos términos $M$ y $N$ que difieren solamente en el nombre de sus variables ligadas se dicen $\alpha$-equivalentes. Esta relación es una relación de equivalencia. Técnicamente, la sustitución está definida sobre clases de $\alpha$-equivalencia de términos


\paragraph{Captura de variables}\label{calculo_lambda:captura_variables} El primer problema se da cuando la sustitución que deseamos realizar sustituye una variable por otra con el mismo nombre que alguna de las variables ligadas de la expresión. Por ejemplo:
$$\replaceBy{(\lambdaAbs{z}{\sigma}{x})}{x}{z} = \lambdaAbs{z}{\sigma}{z}$$

En estos casos, si realizamos la sustitución cambiariamos el significado de la expresión (aquí estariamos convirtiendo la función constante que devuelve $x$ en la función identidad). Por esta razón debemos asegurarnos que cuando realizemos la operación $\replaceBy{(\lambdaAbs{y}{\sigma}{M})}{x}{N}$, la variable ligada $y$ sea renombrada de tal manera que \textbf{no} ocurra libre en $N$.

\vspace*{5mm}
Entonces, teniendo en cuenta lo mencionado, definimos el comportamiento de la operación:

\begin{equation*}
	\begin{split}
		\replaceBy{x}{x}{N} &\equalDef N \\
		\replaceBy{a}{x}{N} &\equalDef a \text{ si } a \in \{true,false\}\cup(\mathcal{X}-\{x\}) \\
		\replaceBy{(\lambdaIf{M}{P}{Q})}{x}{N} &\equalDef \lambdaIf{\replaceBy{M}{x}{N}}{\replaceBy{P}{x}{N}}{\replaceBy{Q}{x}{N}}\\
		\replaceBy{(\lambdaApp{M_1}{M_2})}{x}{N} &\equalDef \lambdaApp{\replaceBy{M_1}{x}{N}}{\replaceBy{M_2}{x}{N}}\\
		\replaceBy{(\lambdaAbs{y}{\sigma}{M})}{x}{N} &\equalDef \lambdaAbs{y}{\sigma}{\replaceBy{M}{x}{N}}~\text{si } x\neq y,~y\notin~FV(N)
	\end{split}
\end{equation*}

La condición $x\neq y,~y\notin~FV(N)$ está para que efectivamente no se produzca la situación mencionada en el parrafo anterior. Y \textbf{siempre} puede cumplirse, solo hay que renombrar las variables de manera apropiada.

\subsubsection{Árbol sintáctico}
Dada una expresión $M$, su árbol sintáctico es un árbol que tiene como raíz a $M$ y como hijos de la raíz a todos los subtérminos válidos de la expresión.
\paragraph{Ejemplos}
El árbol sintáctico de $true$ es:
\begin{center}
	\begin{forest} tikzQtree,
		[$true$]
	\end{forest}
\end{center}

El árbol sintáctico de $\lambdaIf{x}{y}{\lambdaAbs{z}{Bool}{z}}$ es:

\begin{center}
	\begin{forest} tikzQtree,
		[$\lambdaIf{x}{y}{\lambdaAbs{z}{Bool}{z}}$, 
		[$x$]
		[$y$]
		[$\lambdaAbs{z}{Bool}{z}$
		[$z$]
		]
		]
	\end{forest}
\end{center}

\subsection{Sistema de tipado}
El sistema de tipado es un sistema formal de deducción (o derivación) que utiliza axiomas y reglas de tipado para caracterizar un subconjunto de los términos. A estos términos los llamamos \textbf{términos tipados}.

Para que una expresión sea considerada válida dentro de un lenguaje no solo debe ser sintácticamente correcta sino que debemos poder inferir su tipo a través del sistema de tipado que definamos. Si no es posible realizar esta inferencia entonces no la consideraremos una expresión válida del lenguaje.

\paragraph{Contexto de tipado:} Conjunto de pares $x_i:\sigma_i$ que indica los tipos de cada variable de un programa. Por lo general, se nota  $\Gamma = \{x_1:\sigma_1, \dots, x_n:\sigma_n\}$.

\paragraph{Juicio de tipado:} Dado un contexto de tipado $\Gamma$, un \textbf{juicio de tipado} es una expresion $\judgeType{\Gamma}{M}{\sigma}$ que se lee ``el término M tiene tipo $\sigma$ asumiendo el contexto de tipado $\Gamma$''. 

\subsubsection{Axiomas de tipado de \texorpdfstring{$\lambda^b$}{lambda b}}

\begin{equation*}
	\begin{gathered}
		\frac{}{\judgeType{\Gamma}{\lambdaValue{true}}{Bool}}(\text{T-True}) \hspace*{2cm} \frac{}{\judgeType{\Gamma}{\lambdaValue{false}}{Bool}}(\text{T-False})\hspace*{2cm} \\
		\vspace*{5mm} \\
		\frac{x:\sigma\in\Gamma}{\judgeType{\Gamma}{\lambdaValue{x}}{\sigma}}(\text{T-Var})\hspace*{2cm}
	\end{gathered}
\end{equation*}

\vspace*{5mm}

\begin{itemize}
	\item \textbf{T-True} y \textbf{T-False}: $\lambdaValue{true}$ y $\lambdaValue{false}$ son de tipo $Bool$ sin importar el contexto en el que se encuetren.
	\item \textbf{T-Var}: $\lambdaValue{x}$ es de tipo $\sigma$ en $\Gamma$ si el par $\lambdaValue{x}:\sigma$ se encuentra en $\Gamma$.
\end{itemize}
\subsubsection{Reglas de tipado de \texorpdfstring{$\lambda^b$}{lambda b}}
\begin{equation*}
	\begin{gathered}
		\frac{\judgeType{\Gamma}{M}{Bool}\hspace*{5mm}\judgeType{\Gamma}{P}{\sigma}\hspace*{5mm}\judgeType{\Gamma}{Q}{\sigma}}{\judgeType{\Gamma}{\lambdaIf{M}{P}{Q}}{\sigma}}(\text{T-If}) \\
		\vspace*{5mm} \\
		\frac {\judgeType{\Gamma,x:\sigma}{M}{\tau}}
		{\judgeType{\Gamma}{\lambdaValue{\lambdaAbs{x}{\sigma}{M}}}{\sigma\to\tau}}(\text{T-Abs})\hspace*{2cm}
		\frac{\judgeType{\Gamma}{M}{\sigma\to\tau}\hspace*{5mm}\judgeType{\Gamma}{N}{\sigma}}{\judgeType{\Gamma}{\lambdaApp{M}{N}}{\tau}}(\text{T-App})
	\end{gathered}
\end{equation*}

\vspace*{5mm}
\begin{itemize}
	\item \textbf{T-If}: Si $M$ es de tipo $Bool$ y $P$ y $Q$ son de tipo $\sigma$ en $\Gamma$ respecto de $\Gamma$, entonces $\lambdaIf{M}{P}{Q}$ es de tipo $\sigma$ en $\Gamma$.
	\item \textbf{T-Abs}: Si $M$ es de tipo $\tau$ en el contexto $\Gamma\cup \{x:\sigma\}$, entonces $\lambdaAbs{x}{\sigma}{M}$ es de tipo $\sigma\to\tau$ en $\Gamma$.
	\item \textbf{T-App}: Si $M$ es una función que va de $\sigma$ a $\tau$ y $N$ es de tipo $\sigma$ en $\Gamma$, entonces $\lambdaApp{M}{N}$ es de tipo $\tau$.
	
\end{itemize}

\subsubsection{Resultados básicos}

Si $\judgeType{\Gamma}{M}{\sigma}$ puede derivarse usando los axiomas y reglas de tipados decimos que el juicio es \textbf{derivable}. Además, si el juicio se puede derivar para algún $\Gamma$ y $\sigma$, entonces decimos que $M$ es \textbf{tipable}.

\paragraph{Unicidad de tipos} Si $\judgeType{\Gamma}{M}{\sigma}$ y $\judgeType{\Gamma}{M}{\tau}$ son derivables, entonces $\sigma = \tau$. Osea una expresión $M$ puede tener un único tipo dentro de un contexto dado.

\paragraph{Weakening + Strengthening} Si $\judgeType{\Gamma}{M}{\sigma}$ es derivable y $\Gamma\cap\Gamma'$ contiene a todas las variables libres de $M$, entonces $\judgeType{\Gamma'}{M}{\sigma}$.

\paragraph{Sustitución} Si $\judgeType{\Gamma,x:\sigma}{M}{\tau}$ y $\judgeType{\Gamma}{N}{\sigma}$ son derivables, entonces $\judgeType{\Gamma}{\replaceBy{M}{x}{N}}{\tau}$ es derivable.

\subsubsection{Demostración de juicios de tipado}
Dado un sistema tipado, queremos ver si un juicio de tipado es correcto. Para hacer esto, iremos aplicando, al juicio, las reglas del sistema hasta llegar a sus axiomas o hasta llegar a una contradicción o incertidumbre. Si pasa lo primero, entonces el juicio es correcto, si pasa lo segundo, el juicio está mal.

Sea $\Gamma =\{y: Bool\}$. Queremos ver si $\judgeType{\Gamma}{\lambdaIf{true}{false}{y}}{ Bool}$ es derivable. Como es un condicional, debemos fijarnos que cumpla con la regla \textbf{T-If}:

\begin{center}
	\begin{scprooftree}
		\def\extraVskip{5pt}
		
		\AxiomC{$\judgeType{\Gamma}{true}{ Bool}$}
		\AxiomC{$\judgeType{\Gamma}{false}{ Bool}$}
		\AxiomC{$\judgeType{\Gamma}{y}{ Bool}$}
		\RightLabel{T-If}
		\TrinaryInfC{$\judgeType{\Gamma}{\lambdaIf{true}{false}{y}}{ Bool}$}
	\end{scprooftree}
\end{center}

Para que se cumpla esta regla, entonces se debe cumplir cada una de sus condiciones. Dibujamos para cada una de ellas, las reglas que debemos aplicarle:

\begin{center}
	\begin{scprooftree}
		\def\extraVskip{5pt}

\AxiomC{}
\RightLabel{T-True}
\UnaryInfC{$\judgeType{\Gamma}{true}{ Bool}$}

\AxiomC{}
\RightLabel{T-False}
\UnaryInfC{$\judgeType{\Gamma}{false}{ Bool}$}

\AxiomC{$y:Bool \in\Gamma$}
\RightLabel{T-Var}
\UnaryInfC{$\judgeType{\Gamma}{y}{ Bool}$}

\RightLabel{T-If}
		\TrinaryInfC{$\judgeType{\Gamma}{\lambdaIf{true}{false}{y}}{ Bool}$}

	\end{scprooftree}
\end{center}

En el último nivel dibujado, aplicamos \textbf{T-True}, \textbf{T-If} que por ser axiomas y no tener ninguna condición son verdaderas. Además, para la última, aplicamos \textbf{T-Var} que es verdadera porque ocurre $y:Bool \in\Gamma$. Entonces concluimos el juicio de tipado es derivable, es decir la expresión analizada es, efectivamente, de tipo \textit{Bool}

Supongamos ahora que $\Gamma = \phi$, entonces la derivación quedaría:


\begin{center}
	\begin{scprooftree}
		\def\extraVskip{5pt}
		
		\AxiomC{}
		\RightLabel{T-True}
		\UnaryInfC{$\judgeType{\phi}{true}{ Bool}$}
		
		\AxiomC{}
		\RightLabel{T-False}
		\UnaryInfC{$\judgeType{\phi}{false}{ Bool}$}
		
		\AxiomC{\red{$y:Bool \in\phi$}}
		\RightLabel{T-Var}
		\UnaryInfC{$\judgeType{\phi}{y}{ Bool}$}
		
		\RightLabel{T-If}
		\TrinaryInfC{$\judgeType{\phi}{\lambdaIf{true}{false}{y}}{ Bool}$}
		
	\end{scprooftree}
\end{center}

En este caso, sabemos que el par $y:Bool$ no se encuentra en el conjunto vacío, por lo que llegamos a una contradicción. Luego, el juicio no es derivable en este contexto.

\subsection{Semántica operacional}Ya definimos cuales serán los términos y expresiones válidas de nuestro lenguaje. El siguiente paso, es definir algún mecanismo que nos permita inferir el significado o \textbf{valor} de un término. 

\paragraph{Semántica operacional:} Mecanismo que interpreta a los \textbf{términos como estados} de una máquina abstracta y define una \textbf{función de transición} que indica, dado un estado, cual es el siguiente.

De esta forma, el significado de un término $M$ es el estado final que alcanza la máquina al comenzar con $M$ como estado inicial.

Tenemos dos formas de definir la semántica:
\begin{itemize}
	\item \textbf{Small-step}: La función de transición describe un paso de computación. Los términos compuestos se descomponen en términos más simples que deben ser reducidos en un orden específico.
	\item \textbf{Big-step} (o \textbf{Natural Semantics}): La función de transición evalúa el termino a su resultado en un paso.
\end{itemize}

\paragraph{Juicios de evaluación:} Son expresiones de la forma $M\to N$ que se leen ``\textit{el término M reduce, en un paso, al término N}''.

\paragraph{Axiomas de evaluación:} Indican que juicios de evaluación son siempre derivables.

\paragraph{Reglas de evaluación:} Dado un contexto, nos dirán que juicios son derivables.

Las reglas de la semántica asumen que las expresiones están bien tipadas.

\subsubsection{Expresiones Booleanas}\label{calculo_lambda:semantica:booleanas}
Los valores de las  expresiones booleanas son:
$$ V~::=~true~|~false$$
y son usados para reducir el término $\lambdaIf{M_1}{M_2}{M_3}$ mediante los siguientes axiomas:

\begin{equation*}
	\frac{}{\lambdaIf{\lambdaValue{true}}{M_1}{M_2} \to M_1}(\text{E-IfTrue})
\end{equation*}
\vspace*{5mm}
\begin{equation*}
	\frac{}{\lambdaIf{\lambdaValue{false}}{M_1}{M_2} \to M_2}(\text{E-IfFalse})
\end{equation*}
\vspace*{5mm}
\begin{equation*}
	\frac{M_1\to M'_1}{\lambdaIf{M_1}{M_2}{M_3}\to\lambdaIf{M'_1}{M_2}{M_3}}(\text{E-If})
\end{equation*}

\vspace*{5mm}

Si la expresión en la guarda, es un $\lambdaValue{true}$ o un $\lambdaValue{false}$ se reduce la expresión a $M_1$ o $M_2$.

Si $M_1$ es una expresión reducible, entonces se la reduce una y la expresión se remplaza por $\lambdaIf{M'_1}{M_2}{M_3}$.

Con estas reglas definimos la estrategia de evaluación del condicional que se corresponde el orden habitual en lenguajes de programación:
\begin{enumerate}
	\item Primero evaluamos la guarda del condicional
	\item Una vez que la guarda sea un valor, evaluamos la rama que corresponda.
\end{enumerate}

\subsubsection{Propiedades}

\paragraph{Determinismo} Si $M\to M'$ y $M\to M''$ entonces $M' = M''$, esto quiere decir que el valor que no siempre hay una sola regla que nos permite realizar el siguiente paso en la reducción.

\paragraph{Valores en forma normal} Una \textbf{forma normal} es un término que no puede evaluarse más. Consideraremos que terminamos de evaluar un término cuando conseguimos su forma normal.

Todos los valores tiene una forma normal, sin embargo hay que tener en cuenta que como estamos definiendo un lenguaje tipado habrá formas normales que no representen ningún valor.

\subsubsection{Evaluación en muchos pasos}
El juicio de \textbf{evualuación de muchos pasos} $\twoheadrightarrow$ es la clausura reflexiva y transitiva de $\to$. Es decir, la menor relación tal que:
\begin{enumerate}
	\item Si $M\to M'$, entonces $M\twoheadrightarrow M'$
	\item $M\twoheadrightarrow M$ para todo $M$
	\item Si $M\twoheadrightarrow M'$ y $M' \twoheadrightarrow M''$, entonces $M\twoheadrightarrow M''$
\end{enumerate}

\paragraph{Unicidad de formas normales} Si $M\twoheadrightarrow U$ y $M\twoheadrightarrow V$ con $U$ y $V$ formas normales, entonces $U = V$

\paragraph{Terminación}
Para todo $M$ existe una forma normal $N$ tal que $M\twoheadrightarrow N$


\subsection{Semántica operacional de \texorpdfstring{$\lambda^b$}{lambda b}}
En la sección \ref{calculo_lambda:semantica:booleanas} definimos el comportamiento de las expresiones booleanas, sin embargo, nos falta definir como reducir términos del tipo $\lambdaAbs{x}{\sigma}{M}$ y $\lambdaApp{M}{N}$.

Vamos a considerar a los términos de $\lambdaAbs{x}{\sigma}{M}$ como valores, sea $M$ reducible o no. Entonces, nuestro conjunto de valores del lenguaje sería:

$$ V  ~::=~ true~|~false~|~\lambdaAbs{x}{\sigma}{M}$$

Por lo que todo término bien tipado y cerrado (sin variables libres) evalúa a alguna de estos términos. Si es de tipo $Bool$ evalúa a $true$ o $false$, si es de tipo $\sigma\to\tau$ evalúa a $\lambdaAbs{x}{\sigma{M}}$.

A las reglas de reducción agregamos los siguientes ($V$ y $V_1$ son expresiones atómicas):

\begin{equation*}
	\frac{M_1\to M'_1}{\lambdaApp{M_1}{M_2} \to 
		\lambdaApp{M'_1}{M_2}}(\text{E-App1}~ /~ \mu)
\end{equation*}
\vspace*{5mm}
\begin{equation*}
	\frac{M_2 \to M'_2}{\lambdaApp{\lambdaValue{V_1}}{M_2} \to 
		\lambdaApp{\lambdaValue{V_1}}{M'_2}}(\text{E-App2}~/~v)
\end{equation*}	
\vspace*{5mm}
\begin{equation*}
	\frac{}{\lambdaApp{(\lambdaAbs{x}{\sigma}{M})}{\lambdaValue{V}} \to 
		\replaceBy{M}{x}{\lambdaValue{V}}}(\text{E-App2}~/~\beta)
\end{equation*}

\paragraph{Estado de error} Es un estado que \textbf{no es} un valor pero en el que la computación está trabada. Representa el estado en el cual el sistema de runtime de una implementación real generaría una excepción.

El sistema de tipado nos garantiza que si un término cerrado está bien tipado entonces evalúa a un valor.


\paragraph{Corrección}
La corrección de un término nos asegura dos cosas:	\textbf{Progreso} y \textbf{Preservación}.

\begin{itemize}
	\item \textbf{Progreso:} Si $M$ es un término cerrado y bien tipado entonces $M$ es un valor o existe $M'$ tal que $M\to M'$. En otras palabras, la evaluación no puede trabarse para términos cerrados y bien tipados que no son valores. Además, si un programa termina entonces nos devuelve un valor.
	
	\item \textbf{Preservación:} La evaluación de un término $M$ cerrado y bien tipado preserva tipos. Es decir, no importa cuanta veces se reduzca $M$, el término resultante siempre es del tipo original.
	$$\text{Si } \judgeType{\Gamma}{M}{\sigma} \text{ y } M\to N \text{ entonces } \judgeType{\Gamma}{N}{\sigma}$$	
\end{itemize}



\paragraph{Extendiendo el lenguaje}
Para extender el lenguaje debemos realizar los mismos pasos que realizamos para definir el lenguaje $\lambda^b$. Esto es:
\begin{enumerate}
	\item Agregar el nuevo tipo al conjunto de tipos.
	\item Definir los términos de ese tipo.
	\item Definir reglas de tipado y de semántica asegurandonos de que no interfieran con las ya definidas (no deben contradecir, ni crear nuevas formas de inferir algo que ya podía ser inferido). En el apéndice de extensiones hay algunos ejemplos.
\end{enumerate}

\paragraph{Macros} Hay expresiones del lenguaje que usaremos con demasiada frecuencia, a estas expresiones les asignamos macros (\textbf{nombres}) que podremos usar en otras expresiones para evitar escribirlas.

$$Id_{Bool} \equalDef \lambdaAbs{x}{Bool}{x}$$
$$and \equalDef \lambdaAbs{x}{Bool}{\lambdaAbs{y}{Bool}{\lambdaIf{x}{y}{false}}}$$

La expresión $\lambdaApp{Id_{Bool}}{true}$ equivale a $\lambdaApp{(\lambdaAbs{x}{Bool}{x})}{true}$.
\subsection{Extensión Naturales (\texorpdfstring{$\lambda^{bn}$}{lambda bn})}

\paragraph{Tipos}
$$\sigma, \tau ~::=~ Bool~|~Nat~|~\sigma\to\tau$$

\paragraph{Términos}
$$ M~::=~ \dots~|~0~|~succ(M)~|~pred(M)~|~isZero(M) $$

Los términos significan:
\begin{itemize}
	\item $succ(M)$: Sucesor del valor que representa $M$.
	\item $pred(M)$: Predecesor del valor que representa $M$.
	\item $iszero(M)$: Indica si $M$ es equivalente a cero o no.
\end{itemize}

\paragraph{Axiomas y reglas de tipado}
\begin{equation*}
	\frac{}{\judgeType{\Gamma}{0}{Nat}}(\text{T-Zero})
\end{equation*}
\vspace*{5mm}
\begin{equation*}
	\frac{\judgeType{\Gamma}{M}{Nat}}
	{\judgeType{\Gamma}{succ(M)}{Nat}}(\text{T-Succ})\hspace*{1cm}
	\frac{\judgeType{\Gamma}{M}{Nat}}{\judgeType{\Gamma}{pred(M)}{Nat}}(\text{T-Pred})
\end{equation*}
\vspace*{5mm}
\begin{equation*}
	\frac{\judgeType{\Gamma}{M}{Nat}}{\judgeType{\Gamma}{isZero(M)}{Bool}}(\text{T-IsZero})
\end{equation*}

\paragraph{Valores}
$$V~::=~\dots~|~\underline{n}\text{ donde } \underline{n} \text{ abrevia } succ^n(0) \text{ (aplicar $n$ veces $succ$ a $0$) }$$

\paragraph{Axiomas y reglas de evaluación}

\begin{equation*}
		\frac{M_1\to M_1'}{succ(M_1)\to succ(M'_1)}(\text{E-Succ})
	\end{equation*}
	\vspace*{5mm}
	\begin{equation*}
		\frac{}{pred(0)\to 0}(\text{E-PredZero})\hspace*{1cm}\frac{}{pred(succ(\underline{n}))\to \underline{n}}(\text{E-PredSucc})
	\end{equation*}
	\vspace*{5mm}
	\begin{equation*}
		\frac{M_1\to M_1'}{pred(M_1)\to pred(M'_1)}(\text{E-Pred})
	\end{equation*}
	\vspace*{5mm}
	\begin{equation*}
		\frac{}{isZero(0)\to true}(\text{E-IsZeroZero})\hspace*{1cm}\frac{}{isZero(succ(\underline{n}))\to false}(\text{E-isZeroSucc})
	\end{equation*}
	\vspace*{5mm}
	\begin{equation*}
		\hspace*{1cm}\frac{M_1\to M'_1}{isZero(M_1)\to isZero(M_1')}(\text{E-isZero})
	\end{equation*}



\subsection{Simulación de lenguajes imperativos}
Los lenguajes imperativos se caracterizan por su capacidad de asignar y modificar variables dentro de un programa.

\paragraph{Comandos:} Expresiones del lenguaje cuyo objetivo es crear un efecto sobre el estado de la computadora. 

Queremos extender el lenguaje $\lambda$ para que poder simular comandos y efectos sobre la memoria.

\subsubsection{Operaciones básicas}
En un lenguaje imperativo \textbf{todas} las variables son \textbf{mutables}. Definimos tres operaciones básicas para modificar variables:

\begin{itemize}
	\item \textbf{Asignación:} $x := M$ almacena en la referencia $x$ el valor de $M$
	\item \textbf{Alocación (Reserva de memoria)} $ref~M$ genera una referencia fresca cuyo contenido es el valor de $M$
	\item \textbf{Derreferenciación (Lectura):} $!x$ sigue la referencia $x$ y retorna su contenido.
\end{itemize}


Notemos que una vez que agreguemos estas expresiones al lenguaje lambda, este dejará de ser un lenguaje funcional \textbf{puro} (un lenguaje en el todas sus expresiones carecen de efecto).

Nos gustaría agregar las expresiones mencionadas a nuestro lenguaje, para esto primero debemos asignarles un tipo. 

\paragraph{Asignacion} La igualdad ($x := M$) es una expresión de la cual no nos interesa saber su valor sino el efecto que tiene la misma sobre el contexto. Debemos definir un nuevo tipo que nos permita identificar cuando una expresión solo fue evaluada para generar un efecto. Nombraremos este tipo $Unit$ y su conjunto de valores será solo el valor $unit$. Podemos decir que este tipo cumple el rol de $void$ en C.

\subsubsection{Extensión con Referencias (\texorpdfstring{$\lambda^{bnu}$}{lambda bnu})}



\paragraph{Referencias}
Una referencia es una abstracción de una porción de memoria que se encuentra en uso. Usaremos el tipo $Ref~\sigma$ para diferenciar las expresiones que representan referencias. 

\paragraph{Direcciones de memoria:} Representaremos cada posicion con \textbf{direcciones simbólicas} o \textit{locations} usando etiquetas $l,l_1$.

\paragraph{Memoria o \textit{store}:} La definimos como una función parcial $\mu$ que dada una dirección nos devuelve el valor almacenado en ella. Y notaremos:
\begin{itemize}
	\item $\mu[l\to V]$ es el store resultante de \textbf{pisar} $\mu(l)$ con $V$.
	\item $\mu\oplus(l\to V)$ es el \textbf{store extendido} resultante de ampliar $\mu$ con una nueva asociación $l \to V$ asumiendo que $l \notin Dom(\mu)$.    
\end{itemize}

\paragraph{Uso en semántica} Agregaremos las etiquetas al conjunto de valores y, a partir de ahora, los juicios de evaluación, tendrán la siguiente forma: 
$$M|\mu \to M'|\mu'$$
Esto significa que una expresión $M$ reduce a $M'$ y que afecta a $\mu$ de tal forma que pasa a ser $\mu'$, así reflejaremos los cambios de estado de la memoria.

Notemos que, a pesar de que agregamos las etiquetas $l$ como términos y valores, estás son solo producto de la formalización y \textbf{no} se pretende que sean usadas por el programador.

\paragraph{Uso en tipado} Con la posibilidad de modificar la memoria durante la ejecución de un programa, se hace necesaria la definición de un contexto que nos permita inferir el tipo del valor almacenado en las posiciones usadas. Introducimos el \textbf{contexto de tipado} $\Sigma$ para direcciones como una función parcial que toma direcciones y devulve nombres de tipos. Los juicios de tipado serán de la siguiente forma:

$$\judgeType{\Gamma|\Sigma}{M}{\sigma}$$

Indicando que $M$ es de tipo $\sigma$ en el contexto $\Gamma$ cuando el estado de la memoria se corresponde con el contexto de tipado $\Sigma$.

\subsubsection{La extension}

\paragraph{Tipos}
$$\sigma, \tau ~::=~ Bool~|Nat~|~\blue{Unit}~|~\blue{Ref~\sigma}~|~\sigma\to\tau$$

\paragraph{Términos}

$$ M~::=~ \dots~|~unit~|~\lambdaRef{M}~|~!M~|~\lambdaAssign{M}{N}~|~    l$$

\paragraph{Axiomas y reglas de tipado}
\begin{equation*}
	\frac{}{\judgeType{\Gamma|\Sigma}{unit}{Unit}}(\text{T-Unit})\hspace*{1cm}\frac{\judgeType{\Gamma|\Sigma}{M_1}{\sigma}}{\judgeType{\Gamma|\Sigma}{\lambdaRef{M_1}}{Ref~\sigma}}(\text{T-Ref})
\end{equation*}
\vspace*{5mm}
\begin{equation*}
	\frac{\judgeType{\Gamma|\Sigma}{M_1}{Ref~\sigma}}{\judgeType{\Gamma}{!M_1}{\sigma}}(\text{T-DeRef})
\end{equation*}
\vspace*{5mm}
\begin{equation*}
	\frac{\judgeType{\Gamma|\Sigma}{M_1}{Ref~\sigma}\hspace*{5mm}\judgeType{\Gamma|\Sigma}{M_2}{\sigma}}{\judgeType{\Gamma}{\lambdaAssign{M_1}{M_2}}{Unit}}(\text{T-Assing})
\end{equation*}
\vspace*{5mm}
\begin{equation*}
	\frac{\Sigma(l) = \sigma}{\judgeType{\Gamma|\Sigma}{l}{Ref~\sigma}}(\text{T-Loc})
\end{equation*}

\paragraph{Valores}
$$V~::=~\dots~|~unit~|~l$$

\paragraph{Axiomas y reglas semánticas}
\begin{equation*}
	\frac{M_1|\mu\to M'_1|\mu'}{\lambdaApp{M_1}{M_2}|\mu\to \lambdaApp{M'_1}{M_2}|\mu'}(\text{E-App1})\hspace*{1cm}\frac{M_2|\mu\to M'_2|\mu'}{\lambdaApp{\lambdaValue{V_1}}{M_2}|\mu\to \lambdaApp{\lambdaValue{V_1}}{M'_2}|\mu'}(\text{E-App2})
\end{equation*}
\vspace*{5mm}
\begin{equation*}
	\frac{}{(\lambdaApp{\lambdaAbs{x}{\sigma}{M})}{\lambdaValue{V}}|\mu\to \replaceBy{M}{x}{\lambdaValue{V}}|\mu}(\text{E-AppAbs})
\end{equation*}
\vspace*{5mm}
\begin{equation*}
	\frac{M_1|\mu\to M'_1|\mu'}{!M_1|\mu\to !M_1'|\mu'}(\text{E-DeRef})\hspace*{1cm}
	\frac{\mu(l) = \lambdaValue{V}}{!l|\mu\to V|\mu}(\text{E-DerefLoc})
\end{equation*}
\vspace*{5mm}
\begin{equation*}
	\frac{M_1|\mu\to M'_1|\mu'}{\lambdaAssign{M_1}{M_2}|\mu\to \lambdaAssign{M'_1}{M_2}|\mu'}(\text{E-Assign1})
\end{equation*}
\vspace*{5mm}
\begin{equation*}
	\frac{M_2|\mu\to M'_2|\mu'}{\lambdaAssign{\lambdaValue{V}}{M_2}|\mu\to \lambdaAssign{\lambdaValue{V}}{M'_2}|\mu'}(\text{E-Assign2})
\end{equation*}
\vspace*{5mm}
\begin{equation*}
	\frac{}{\lambdaAssign{l}{\lambdaValue{V}}|\mu\to unit|\mu[l\to \lambdaValue{V}]}(\text{E-Assign})
\end{equation*}
\vspace*{5mm}
\begin{equation*}
	\frac{M_1|\mu\to M'_1|\mu'}{\lambdaRef{M_1}|\mu\to \lambdaRef{M'_1}|\mu'}(\text{E-Ref})\hspace*{1cm}
	\frac{l\notin Dom(\mu)}{\lambdaRef{\lambdaValue{V}}|\mu\to l|\mu\oplus(l\to \lambdaValue{V})}(\text{E-RefV})
\end{equation*}

\paragraph{Macro punto y coma ( ; )} En lenguajes con efectos laterales, como el que estamos definiendo, esta macro nos servirá para definir el orden de evaluación de varias expresiones en \textbf{secuencia}.

$$M_1;M_2 \equalDef \lambdaApp{(\lambdaAbs{x}{Unit}{M_2})}{M_1} \hspace*{5mm} x\notin FV(M_2)$$

Por como definimos las reglas semánticas del lenguaje, esto significa que primero se evalúa $M_1$ y luego $M_2$. 


\subsubsection{Correción de tipos en un lenguaje con referencias}
Debemos reformular las definiciones de corrección del lenguaje, es decir, debemos indicar que significa el \textbf{progreso} y la \textbf{preservación} cuando hay referencias.

\subsubsection{Preservación} La preservación nos aseguraba que no importa cuantas veces reduzcamos una expresión, esta debería mantener su tipo. Con las asignaciones podemos cambiar el tipo de ciertos valores durante la ejecución de un programa, lo que implica que la expresión podria cambiar su tipo. Para definir la preservación precisamos una noción de compatibilidad entre el store y el contexto de tipado que nos permita asegurar que si los valores no cambian su tipo, entonces la expresión lo mantiene.

\paragraph{Compatibilidad:}
Diremos que $\mu$ es compatible con $\Sigma$ ($\Gamma|\Sigma\triangleright\mu$) si ambas funciones tiene el mismo dominio, y los tipos de cada etiqueta de $\mu$ coinciden con los tipos que se les asignó en $\Sigma$.

$$ \Gamma|\Sigma\triangleright\mu  \iff Dom(\Sigma) = Dom(\mu) \land (\forall~\mu)~\judgeType{\Gamma|\Sigma}{\mu(l)}{\Sigma(l)}$$

\paragraph{Preservación:} Dada una expresión $M$ tal que
\begin{itemize}
	\item $\judgeType{\Gamma|\Sigma}{M}{\sigma}$, 
	\item $M|\mu\to N|\mu'$ y
	\item $\Gamma|\Sigma\triangleright\mu$
\end{itemize}
entonces existe un 	$\Sigma'$ que contiene a $\Sigma$ tal que:
\begin{itemize}
	\item $\judgeType{\Gamma|\Sigma'}{N}{\sigma}$ y
	\item $\Gamma|\Sigma'\triangleright\mu'$
\end{itemize}

Para que $\mu'$ sea compatible con $\Sigma'$ puede haber dos posibilidad: O qué $\mu' = \mu$ y $\Sigma = \Sigma'$ o qué $\mu'$ sea una extensión de $\mu$, es decir que se haya creado una referencia nueva, en cuyo caso ninguno de los tipos fue modificado y $\Sigma'$ es $\Sigma$ extendido con el tipo de la nueva referencia.

\subsubsection{Progreso} El progreso asegura que dada una expresión, su ejecución termina en un valor o no termina. Para el nuevo lenguaje, hay que tener en cuenta el contexto de tipado:

\begin{centrado}
	Si $M$ es cerrado y bien tipado en un contexto de tipado de memoria $\Sigma$, entonces
	\begin{itemize}
		\item $M$ es un valor
		\item o bien para cualquier memoria $\mu$ que sea compatible con $\Sigma$, existe $M'$ y $\mu'$ tal que $M|\mu\to M'|\mu'$
	\end{itemize}
\end{centrado}

Esto quiere decir que solo se puede asegurar progreso cuando $\mu$ es compatible con $\Sigma$.

\subsection{Extensión con recursión}\label{lambda_calculo:recursion}
Queremos dar al lenguaje $\lambda$ la capacidad de interpretar expresiones recursivas. 
Definimos la función $fix~M$ que, dada una función $M$, devuelve su punto fijo, es decir, un valor $x$ tal que $M~x$ evalúa a $x$.

\paragraph{Términos}
$$M~:=~\dots~|~\lambdaFix{M}$$

\paragraph{Regla de tipado}
\begin{equation*}
	\frac{\judgeType{\Gamma}{M}{\sigma\to\sigma}}{\judgeType{\Gamma}{\lambdaFix{M}}{\sigma}}(\text{T-Fix})
\end{equation*}

\paragraph{Reglas de evaluación}
\begin{equation*}
	\frac{M_1\to M'_1}{\lambdaFix{M_1}\to\lambdaFix{M'_1}}(\text{E-Fix})
\end{equation*}
\vspace*{5mm}
\begin{equation*}
	\frac{}{\lambdaFix{(\lambdaAbs{x}{\sigma}{M})}\to\replaceBy{M}{x}{\lambdaFix{\lambdaAbs{x}{\sigma}{M}}}}(\text{E-FixBeta})
\end{equation*}

\paragraph{Ejemplo:} Queremos expresar la función factorial:
$$f(n) = \textbf{If } n=0 \textbf{ then } 1 \textbf{ else } n*f(n-1)$$

Lo logramos con:
$$fact = \lambdaFix{\lambdaAbs{f}{Nat\rightarrow Nat}{(\lambdaAbs{x}{\text{Nat}}{\lambdaIf{isZero(x)}{1}{x*(\lambdaApp{f}{pred(x)})}})}}$$

De adentro hacia afuera:
\begin{itemize}
	\item La abstracción $\lambda x$ toma un natural y devuelve un natural. Si $x$ es cero, entonces devuelve $1$, sino $x*(f~pred(x))$. En esta expresión $f$ es una variable libre, puede ser cualquier cosa.
	\item La abstracción $\lambda f$ liga $f$ a una función $Nat\to Nat$ y nos permite \textit{parametrizarla}. Osea que $f$ podría ser, por ejemplo, $(\lambdaAbs{y}{Nat}{succ(y)})$ y en ese caso estariamos devolviendo el mismo $x$.
	\item Por último $fix$, nos dice que la función $f$ debe ser $fact$.
\end{itemize}

Notemos que $fact$ es una abstracción que toma un $Nat$ y devuelve un $Nat$. Esto es porque $fix$ \textit{aplica} $fact$ a $\lambda f$, lo que devuelve $\lambda x$ con $f$ remplazado por $fact$.

\newpage

\appendix
\chapter{Programación funcional en Haskell}
\paragraph{Tipos elementales}
\begin{centrado}
	\begin{minted}{haskell}
1               -- Int          Enteros
'a'             -- Char         Caracteres
1.2             -- Float        Números de punto flotante
True            -- Bool         Booleanos
[1,2,3]         -- [Int]        Listas
(1, True)       -- (Int, Bool)  Tuplas, pares
length          -- [a] -> Int   Funciones
length [1,2,3]  -- Int          Expresiones
\x -> x         -- a -> a       Funciones anónimas
	\end{minted}
\end{centrado}

\paragraph{Guardas}
\begin{centrado}
	\begin{minted}{haskell}
signo n | n >= 0    = True
        | otherwise = False
	\end{minted}
\end{centrado}

\paragraph{Pattern Matching}
\begin{centrado}
	\begin{minted}{haskell}
longitud [] = 0
longitud (x:xs) = 1 + (longitud xs)
	\end{minted}
\end{centrado}

\paragraph{Polimorfismo paramétrico}
\begin{centrado}
	\begin{minted}{haskell}
todosIguales :: Eq a => [a] -> Bool
todosIguales [] = True
todosIguales [x] = True
todosIguales (x:y:xs) = x == y && todosIguales(y:xs)
	\end{minted}
\end{centrado}

\paragraph{Clases de tipo}
\begin{centrado}
	\begin{minted}{haskell}
Eq a    -- Tipos con comparación de igualdad
Num a   -- Tipos que se comportan como los números
Ord a   -- Tipos orden
Show a  -- Tipos que pueden ser representados como strings
	\end{minted}
\end{centrado}

\paragraph{Definición de listas}
\begin{centrado}
	\begin{minted}[breaklines]{haskell}
[1,2,3,4,5]                 -- Por extensión
[1 .. 4]                    -- Secuencias aritméticas
[ x | x <- [1..], esPar x ] -- Por compresión

-- Las listas pueden ser infinitas, solo hay que tener cuidado cuando las usamos. Ejemplo de lista infinita:
	
infinitosUnos :: [Int]
infinitosUnos = 1 : infinitosUnos

puntosDelCuadrante :: [(Int, Int)]
puntosDelCuadrante = [ (x, s-x) | s <- [0..], x <-[0..s] ]
	\end{minted}
\end{centrado}

\paragraph{Funciones de alto orden}
\begin{centrado}
	\begin{minted}[breaklines]{haskell}
mejorSegun :: (a -> a -> Bool) -> [a] -> a
mejorSegun _ [x] = x
mejorSegun f (x : xs) | f x (mejorSegun f xs) = x
                      | otherwise = mejorSegun f xs
\end{minted}
\end{centrado}

\section{Otros tipos útiles}
\paragraph{Formula}
\begin{centrado}
	\begin{minted}[breaklines]{haskell}
data Formula = Proposicion String | No Formula 
                                  | Y Formula Formula
                                  | O Formula Formula
                                  | Imp Formula Formula
                                  
foldFormula :: (String -> a) -> (Formula -> a) -> 
               (Formula -> Formula -> a) -> (Formula -> Formula -> a) 
               -> (Formula -> Formula -> a) -> Formula -> a
foldFormula fp fn fy fo fImp form = case form of :
		Proposicion s -> fp s
		No sf -> fn (rec sf)
		Y sf1 sf2 -> fy (rec sf1) (rec sf2)
		O sf1 sf2 -> fo (rec sf1) (rec sf2)
		Impl sf1 sf2 -> fImpl (rec sf1) (rec sf2)
	where rec = foldForm fp fn fy fo fImp
	\end{minted}
	\end{centrado}

\paragraph{Rosetree}
\begin{centrado}
	\begin{minted}[breaklines]{haskell}
data Rosetree = Rose a [Rosetree]
-- Hay varias formas de definir el fold para esta estructura
foldRose :: (a -> [b] -> b) -> Rosetree a -> b
foldRose f ( Rose x l ) = f x ( map ( foldRose f ) l )
	
foldRose2 :: ( a -> c -> b) -> ( b -> c -> c ) -> c 
            -> Rosetree a -> b
foldRose2 g f z (Rose x l) = 
          g x ( foldr f z ( map ( foldRose g f z ) l ) )
		
\end{minted}
\end{centrado}


\newpage
\chapter{Extensiones del lenguaje \texorpdfstring{$\lambda^b$}{lambda b}}



\section{Registros \texorpdfstring{$\lambda^{...r}$}{lambda ...r}}

\paragraph{Tipos}
$$\sigma, \tau ~::=~...~|~\{l_i : \sigma_i ~^{i\in 1..n}\}$$

El tipo $\{l_i : \sigma_i^{i\in 1..n}\}$ representan las estructuras con $n$ atributos tipados, por ejemplo: $\{nombre : String,edad:Nat\}$
\paragraph{Términos}
$$ M~::=~ \dots~|~\{l_i = M_i ~^{i\in 1..n}\}~|~M.l $$

Los términos significan:
\begin{itemize}
    \item El registro $\{l_i = M_i ~^{i\in 1..n}\}$ evalua $\{l_i = V_i ~^{i\in 1..n}\}$  donde $V_i$ es el s al que evalúa $M_i$ para $i\in 1..n$.
    \item $M.l$: Proyecta el valor de la etiqueta $l$ del registro $M$
\end{itemize}

\paragraph{Axiomas y reglas de tipado}
\begin{equation*}
\frac{\judgeType{\Gamma}{M_i}{\sigma_i} \text{ para cada } i \in 1..n}{\judgeType{\Gamma}{\{l_i = M_i ~^{i\in 1..n}\}}{\{l_i : \sigma_i ~^{i\in 1..n}\}}}(\text{T-RCD})
\end{equation*}
\vspace*{5mm}
\begin{equation*}
\frac{\judgeType{\Gamma}{\{l_i = M_i ~^{i\in 1..n}\}}{\{l_i : \sigma_i ~^{i\in 1..n}\}}\hspace*{5mm} j \in 1..n}
{\judgeType{\Gamma}{M.l_j}{\sigma_j}}(\text{T-Proj})
\end{equation*}

\paragraph{Valores}
$$V~::=~\dots~|~\{l_i = V_i ~^{i\in 1..n}\}$$

\paragraph{Axiomas y reglas de evaluación}

\begin{equation*}
\frac{j\in 1..n}{\{l_i = \lambdaValue{V_i} ~^{i\in 1..n}\}.l_j \to \lambdaValue{V_j}}(\text{E-ProjRcd})
\end{equation*}
\vspace*{5mm}
\begin{equation*}
\frac{M \to M'}{M.l \to M'.l}(\text{E-Proj})
\end{equation*}

\vspace*{5mm}
\begin{equation*}
\frac{M_j\to M_j'}{\{l_i = \lambdaValue{V_i}~^{i\in 1..j-1}, l_j = M_j, l_i = M_i ~^{i\in j+1..n}\} \to \{l_i = \lambdaValue{V_i}~^{i\in 1..j-1}, l_j = M'_j, l_i = M_i ~^{i\in j+1..n}\}}(\text{E-RCD})
\end{equation*}
\vspace*{5mm}
\section{Declaraciones Locales (\texorpdfstring{$\lambda^{...let}$}{lambda ...let})}\label{extension_lambda:let}

Con esta extensión, agregamos al lenguaje el término $\lambdaLet{x}{\sigma}{M}{N}$, que evalúa $M$ a un valor, liga $x$ a $V$ y, luego, evalúa $N$. Este término solo mejora la legibilidad de los programas que ya podemos definir con el lenguaje hasta ahora definido.

\paragraph{Términos}
$$ M~::=~ \dots~|~\lambdaLet{x}{\sigma}{M}{N} $$


\paragraph{Axiomas y reglas de tipado}
\begin{equation*}
\frac{\judgeType{\Gamma}{M}{\sigma_1}\hspace*{5mm}\judgeType{\Gamma,x:\sigma_1}{N}{\sigma_2}}{\judgeType{\Gamma}{\lambdaLet{x}{\sigma_1}{M}{N}}{\sigma_2}}(\text{T-Let})
\end{equation*}

\paragraph{Axiomas y reglas de evaluación}

\begin{equation*}
\frac{M_1\to M_1'}{\lambdaLet{x}{\sigma}{M_1}{M_2}\to \lambdaLet{x}{\sigma}{M'_1}{M_2}}(\text{E-Let})
\end{equation*}
\vspace*{5mm}
\begin{equation*}
\frac{}{\lambdaLet{x}{\sigma}{\lambdaValue{V_1}}{M_2}\to \replaceBy{M_2}{x}{\lambdaValue{V_1}}}(\text{E-LetV})
\end{equation*}

\subsubsection{Construcción \textit{let} recursivo (Letrec)}
Una construcción alternativa para definir funciones recursivas es 
$$letrec~f:\sigma\to\sigma = \lambdaAbs{x}{\sigma}{M~in~N}$$

Y $letRec$ se puede definir  en base a $let$ y $fix$ (definido en \ref{lambda_calculo:recursion}) de la siguiente forma:

$$\lambdaLet{f}{\sigma\to\sigma}{(\lambdaFix{\lambdaAbs{f}{\sigma\to\sigma}{\lambdaAbs{x}{\sigma}{M}}})}{N}$$

\section{Tuplas}

\paragraph{Tipos}
$$\sigma,\tau~::= \dots~|~\sigma\times\tau$$

\paragraph{Términos}
$$M,~N~::=~\dots~|~<M,N>~|~\pi_1(M)~|~\pi_2(M)$$
\paragraph{Axiomas y reglas de tipado}
\begin{equation*}
    \frac{\judgeType{\Gamma}{M}{\sigma}\hspace*{5mm}\judgeType{\Gamma}{N}{\tau}}{\judgeType{\Gamma}{<M,N>}{\sigma\times\tau}}(\text{T-Tupla})
\end{equation*}
\vspace*{5mm}
\begin{equation*}
\frac{\judgeType{\Gamma}{M}{\sigma\times\tau}}{\judgeType{\Gamma}{\pi_1(M)}{\sigma}}(\text{T-}\pi_1)\hspace*{1cm}\frac{\judgeType{\Gamma}{M}{\sigma\times\tau}}{\judgeType{\Gamma}{\pi_2(M)}{\tau}}(\text{T-}\pi_2)
\end{equation*}

\paragraph{Valores}
$$V~::=~\dots~|~<V,V>$$

\paragraph{Axiomas y reglas de evaluación}
\begin{equation*}
\frac{M\to M'}{<M,N>\to<M',N>}(\text{E-Tuplas})\hspace*{1cm}\frac{N\to N'}{<\lambdaValue{V},N>\to<\lambdaValue{V},N'>}(\text{E-Tuplas1})
\end{equation*}
\vspace*{5mm}
\begin{equation*}
\frac{M\to M'}{\pi_1(M)\to\pi_1(M')}(\text{E-}\pi_1)\hspace*{1cm}\frac{}{\pi_1(<\lambdaValue{V_1}, \lambdaValue{V_2}>)\to\lambdaValue{V_1}}(\text{E-}\pi'_1)
\end{equation*}
\vspace*{5mm}
\begin{equation*}
\frac{M\to M'}{\pi_2(M)\to\pi_2(M')}(\text{E-}\pi_2)\hspace*{1cm}\frac{}{\pi_2(<\lambdaValue{V_1}, \lambdaValue{V_2}>)\to\lambdaValue{V_2}}(\text{E-}\pi'_2)
\end{equation*}

\section{Árboles binarios}

\paragraph{Tipos}
$$\sigma,\tau~::= \dots~|~AB_\sigma$$

\paragraph{Términos}
$$M,~N~::=~\dots~|~\text{Nil}_\sigma~|~\text{Bin}(M, N, O)~|~\text{raiz}(M)~|~\text{der}(M)~|~\text{izq}(M)~|~\text{esNil}(M)$$
\paragraph{Axiomas y reglas de tipado}
\begin{equation*}
\begin{gathered}
    \frac{}{\judgeType{\Gamma}{\text{Nil}_\sigma}{AB_\sigma}}(\text{T-Nil})\hspace*{1cm}
\frac{\judgeType{\Gamma}{M}{AB_\sigma}\hspace*{5mm}\judgeType{\Gamma}{N}{\sigma}\hspace*{5mm}\judgeType{\Gamma}{O}{AB_\sigma}}{\judgeType{\Gamma}{\text{Bin}(M, N, O)}{AB_\sigma}}(\text{T-Bin}) \\
\vspace*{5mm}\\
\frac{\judgeType{\Gamma}{M}{AB_\sigma}}{\judgeType{\Gamma}{\text{raiz}(M)}{\sigma}}(\text{T-raiz})\hspace*{1cm}
\frac{\judgeType{\Gamma}{M}{AB_\sigma}}{\judgeType{\Gamma}{\text{der}(M)}{AB_\sigma}}(\text{T-der})
\vspace*{5mm} \\
\frac{\judgeType{\Gamma}{M}{AB_\sigma}}{\judgeType{\Gamma}{\text{izq}(M)}{AB_\sigma}}(\text{T-izq})
\hspace*{1cm}
\frac{\judgeType{\Gamma}{M}{AB_\sigma}}{\judgeType{\Gamma}{\text{isNil}(M)}{Bool}}(\text{T-isNil})
\end{gathered}
\end{equation*}

\paragraph{Valores}
$$V~::=~\dots~|~\text{Nil}~|~\text{Bin}(V,V,V)$$

\paragraph{Axiomas y reglas de evaluación}
\begin{equation*}
\frac{M\to M'}{\text{Bin}(M,N,O)\to \text{Bin}(M',N,O)}(\text{E-Bin1})\hspace*{1cm}\frac{N\to N'}{\text{Bin}(V,N,O)\to \text{Bin}(V,N',O)}(\text{E-Bin2})
\end{equation*}
\vspace*{5mm}
\begin{equation*}
\frac{O\to O'}{\text{Bin}(V_1,V_2,O)\to \text{Bin}(V_1,V_2,O')}(\text{E-Bin3})
\end{equation*}
\vspace*{5mm}
\begin{equation*}
\frac{M\to M'}{\text{raiz}(M)\to\text{raiz}(M')}(\text{E-Raiz1})\hspace*{1cm}\frac{}{\text{raiz}(\text{Bin}(V_1,V_2,V_3))\to V_2}(\text{E-Bin3})
\end{equation*}
\vspace*{5mm}
\begin{equation*}
\frac{M\to M'}{\text{der}(M)\to\text{der}(M')}(\text{E-Der1})\hspace*{1cm}\frac{}{\text{der}(\text{Bin}(V_1,V_2,V_3))\to V_3}(\text{E-Der2})
\end{equation*}
\vspace*{5mm}
\begin{equation*}
\frac{M\to M'}{\text{izq}(M)\to\text{izq}(M')}(\text{E-Izq1})\hspace*{1cm}\frac{}{\text{izq}(\text{Bin}(V_1,V_2,V_3))\to V_1}(\text{E-Izq2})
\end{equation*}
\hspace*{5mm}
\begin{equation*}
\frac{}{\text{isNil}(M)\to\text{izq}(M')}(\text{E-isNil1})\hspace*{1cm}\frac{}{\text{isNil}(\text{Bin}(V_1,V_2,V_3))\to false}(\text{E-isNilBin})
\end{equation*}
\hspace*{5mm}
\begin{equation*}
\frac{}{\text{isNil}(\text{Bin}(V_1,V_2,V_3))\to true}(\text{E-isNilNil})
\end{equation*}
\end{document}