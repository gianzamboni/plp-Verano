\documentclass[10pt,a4paper]{article}

\usepackage[spanish]{babel}
\usepackage[utf8]{inputenc}

 
\usepackage[pdftex,
pdfauthor={Gianfranco Zamboni},
pdftitle={Paradigmas de Lenguajes de Programación},
pdfsubject={},
pdfkeywords={Resumen , Computacion, FCEyN, UBA, Paradigmas de Lenguajes de Programación, Imperativo, Funcional, Cálculo Lambda, Programación Orientada a Objetos, Objetos, Programación Lógica, Recursion, Tipado, Sintaxis, Subtipado, Semantica, Programacion orientada a objetos},
pdfproducer={Latex with hyperref},
pdfcreator={pdflatex},
pdfencoding = auto]{hyperref}

\usepackage{a4wide}
\usepackage{amsmath}
\usepackage{amssymb}
\usepackage{fancyhdr}
\usepackage{forest}
\usepackage{tikz}
\usepackage{minted}
\usepackage{multicol}
\usepackage{mathabx}
\usepackage{bussproofs}

\pagestyle{fancy}
\fancyhf{}
\renewcommand{\sectionmark}[1]{\markright{\thesection\ - #1}}
\fancyhead[LO]{Sección \rightmark }  
\fancyhead[RO]{\small{Paradigmas de Lenguajes de Programación}}
\fancyfoot[CO]{\thepage}
\renewcommand{\headrulewidth}{0.5pt}
\renewcommand{\footrulewidth}{0.5pt}
\setlength{\headsep}{1cm}
\setlength{\headheight}{13.07225pt}

\renewcommand{\baselinestretch}{1.2}  % line 

\setcounter{tocdepth}{2}% Allow only \chapter in ToC


\newenvironment{centrado}
    {
     \begin{center}
     \begin{minipage}{0.8\textwidth}
 }    
    {
     \end{minipage}
     \end{center}
    }

\newcommand{\rel}{\ensuremath{\mathcal{R}}}

\newcommand{\equalDef}{\overset{def}{=}}
\newcommand{\equalDot}{\overset{\cdot}{=}}

\newcommand{\lambdaAbs}[3]{\lambda #1: #2 . #3}
\newcommand{\lambdaAssign}[2]{#1~:=~#2}
\newcommand{\lambdaApp}[2]{#1~#2}
\newcommand{\lambdaIf}[3]{if~ #1~ then~ #2~ else~ #3}
\newcommand{\lambdaTrue}{true}
\newcommand{\lambdaFalse}{false}
\newcommand{\lambdaLet}[4]{let~#1:#2 = #3~in~#4}
\newcommand{\lambdaRef}[1]{ref~#1}
\newcommand{\lambdaVar}[1]{#1}
\newcommand{\lambdaValue}[1]{\color{red}#1\color{black}}
\newcommand{\lambdaFix}[1]{fix~#1}

\newcommand{\lambdaAbsI}[2]{\lambda #1. #2}
\newcommand{\lambdaLetI}[3]{let~#1 = #2~in~#3}



\newcommand{\blue}[1]{\color{blue}#1\color{black}}
\newcommand{\replaceBy}[3]{#1\{#2\leftarrow#3\}}

\newcommand{\judgeType}[3]{#1\triangleright #2 : #3}
\newcommand{\judgeTypeS}[3]{#1\mapsto #2 : #3}


\newcommand{\OOAtributo}[3]{#1 = \varsigma(#2)#3}
\newcommand{\OORedefinicion}[3]{#1 \leftleftharpoons \varsigma(#2)#3}
\newcommand{\OOReduccion}[2]{#1 \longrightarrow #2}
\newcommand{\OORep}[1]{\green{[[}#1\green{]]}}

\newenvironment{scprooftree}
{\leavevmode\hbox\bgroup}
{\DisplayProof\egroup}

\tikzset{
    every leaf node/.style={text=red, align=center},
    every tree node/.style={text=blue, align=center},
}

\forestset{tikzQtree/.style={for tree={if n children=0{
                node options=every leaf node/.try}{node options=every tree node/.try}, text centered}}}
                
                
\DeclareMathOperator{\Erase}{Erase}

\newcommand{\List}[1]{[~]_{#1}}
\newcommand{\lambdaListCase}[3]{\text{case } #1 \text{ of } \{[~]\leadsto#2~|~h::t\leadsto#3\}}

\newcommand{\WFunc}{\mathbb{W}}

\newcommand{\red}[1]{{\color{red}#1}}%\renewcommand{\appendixtocname}{Apéndices}
\newcommand{\green}[1]{{\color{green!60!black}#1}}%\renewcommand{\appendixpagename}{Apéndices}





\begin{document}
\title{Resumen: Paradigmas de Lenguajes de Programación}

\date{\today}

\author{Zamboni, Gianfranco}

\section{Subtipado}

\begin{align*}
\frac{}{Nat <: Float}(\text{S-NatFloat}) \hspace*{5mm}\frac{}{Int <: Float}(\text{S-IntFloat}) \hspace*{5mm}\frac{}{Bool <: Nat}(\text{S-BoolNat})
\end{align*}

$$\frac{\sigma' <: \sigma\hspace*{5mm}\tau <: \tau'}{\sigma\to\tau <: \sigma'\to\tau'}(\text{S-Func})$$

$$\frac{}{\sigma <: \sigma}(\text{S-Refl}) \hspace*{1cm}\frac{\sigma <:\tau\hspace*{5mm}\tau <: \rho}{\sigma <: \rho}(\text{S-Trans})$$

$$\frac{\sigma <: \tau\hspace*{5mm} \tau <: \sigma}{Ref~\tau <: Ref~\sigma}$$

\begin{align*}
\frac{\sigma <: \tau}{Source~\sigma <: Source~\tau}(\text{S-Source})\hspace*{1cm}\frac{\tau <: \sigma}{Sink~\sigma <: Sink~\tau}(\text{S-Sink})
\end{align*}

\begin{align*}
\frac{}{Ref~\tau <: Source~\tau}(\text{S-RefSource})\hspace*{1cm}\frac{}{Ref~\tau <: Sink~\tau}(\text{S-RefSink})
\end{align*}
\subsection{Reglas de reduccion con subtipado}

\begin{equation*}
\begin{gathered}
    \frac{x:\sigma\in\Gamma}{\judgeTypeS{\Gamma}{x}{\sigma}}(\text{T-Var})\hspace*{2cm}
\vspace*{5mm} \\
    \frac {\judgeTypeS{\Gamma,x:\sigma}{M}{\tau}}
          {\judgeTypeS{\Gamma}{\lambdaAbs{x}{\sigma}{M}}{\sigma\to\tau}}(\text{T-Abs})\hspace*{2cm}
    \frac{\judgeTypeS{\Gamma}{M}{\sigma\to\tau}\hspace*{5mm}\judgeTypeS{\Gamma}{N}{\blue{\rho}}\hspace*{5mm}\blue{\rho <: \sigma}}{\judgeTypeS{\Gamma}{M~N}{\tau}}(\text{T-App})
\end{gathered}
\end{equation*}

\newpage
\section{Objetos}
\subsubsection{Sintaxis}
\begin{tabular}{lllll}
	$a,b$ &$::=$& &$x$ & Variables \\
 	      &     & $|$ &$[\OOAtributo{l_i}{x_i}{b_i}^{i\in1..n}]$ &  Objetos\\
 	      &     & $|$ &$a.l$ &  Selección/ Envío de mensajes \\
 	      &     & $|$ &$\OORedefinicion{a.l}{x}{b}$ &  Redefinición de un método.	       	       	      
\end{tabular}

\subsection{Variables libres}
\begin{tabular}{ll}
	$\text{fv}(\varsigma(x)b)$ &$= \text{fv}(b)\backslash \{x\} $\\
	$\text{fv}(x)$ &$= \{x\} $\\
	$\text{fv}([\OOAtributo{l_i}{x_i}{b_i}^{i\in 1..n}])$ &$=  \bigcup^{1\in 1..n} \text{fv}(\varsigma(x)b)$\\
	$\text{fv}(a.l)$ &$= \text{fv}(a) $\\
	$\text{fv}(\OORedefinicion{a.l}{x}{b})$ &$= \text{fv}(a.l)\cup \text{fv}(\varsigma(x)b) $\\
\end{tabular}

\subsection{Sustitución}
\begin{center}
\begin{tabular}{lll}
	$x\{x \leftarrow c\}$ &$= c$ & \\
	$y\{x \leftarrow c\}$ &$= y$ & si $x\neq y$\\
	$([\OOAtributo{l_i}{x_i}{b_i}^{i\in 1..n}])\{x \leftarrow c\}$ &$=  [l_i = (\varsigma(x_i)b_i)\{x \leftarrow c\}^{i\in 1..n}]$ & \\
	$(a.l)\{x \leftarrow c\}$ &$= (a\{x \leftarrow c\}).l $ & \\
	$(\OORedefinicion{a.l}{x}{b})\{x \leftarrow c\}$ &$= (a\{x \leftarrow c\}).l \leftleftharpoons (\varsigma(x)b)\{x \leftarrow c\} $ & \\
	$(\varsigma(y)b)\{x \leftarrow c\}$ &$= (\varsigma(y')(b\{y \leftarrow y'\}\{x \leftarrow c\})) $ & si $y'\notin$fv$(\varsigma(y)b)\cup$fv$(c)\cup\{x\}$ \\
\end{tabular}
\end{center}

\subsection{Semantica operacional}
$$V~::=~[\OOAtributo{l_i}{x_i}{b_i}^{1\in 1..n}]$$

$$\frac{}{v\longrightarrow v}[\text{Obj}]$$
\vspace*{5mm}
$$\frac{a\longrightarrow v'\hspace*{5mm} v'\equiv [\OOAtributo{l_i}{x_i}{b_i}^{i\in 1..n}]\hspace*{5mm} b_j\{x_j\leftarrow v'\}\longrightarrow v\hspace*{5mm} j\in1..n}{a.l_j\longrightarrow v}[\text{Sel}]$$

\vspace*{5mm}
$$\frac{a\longrightarrow [\OOAtributo{l_i}{x_i}{b_i}^{i\in 1..n}]\hspace*{5mm} j\in1..n}{\OORedefinicion{a.l_j}{x}{b}\longrightarrow [\OOAtributo{l_j}{x}{b},~\OOAtributo{l_i}{x_i}{b_i}^{i\in 1..n-\{j\}}]}[\text{Upd}]$$

\paragraph{Indefinido: } $[\OOAtributo{a}{x}{x.a}].a$

\subsubsection{Codificacion de funciones}
\begin{align*}
\OORep{x} &\equalDef x\\
\OORep{M~N} &\equalDef \OORep{M}.arg :=~\OORep{N}\\
\OORep{\lambdaAbsI{x}{M}} &\equalDef 
[\OOAtributo{val}{y}{\OORep{M}\{x\leftarrow y.arg\}},~\OOAtributo{arg}{y}{y.arg}]\\
\end{align*}


\newpage
\section{Resolución}
\subsection{Lógica propocisional}
$$\frac{C_1 = \{A_1,\dots,A_m,L\}\hspace*{5mm} C_2 = \{B_1,\dots,B_m, \overline{L}\}}{C = \{A_1,\dots,A_m, B_1,\dots,B_n\}}$$

\subsection{Lógica de primer orden}
Transformar la formula:
\begin{enumerate}
\item Eliminar las implicaciones, es decir, si aparece una clausula de la forma $(A\supset B)$, reescribirla como $(\lnot A \lor B)$.
\item Pasar a \textbf{forma normal negada}.
\item Pasar a \textbf{forma normal prenexa}.
\item Pasar a \textbf{forma normal de Skolem}.
\item Pasar a \textbf{forma normal conjuntiva}.
\item \textbf{Distribuir} cuantificadores universales.
\end{enumerate}

\subsubsection{Skolemización}

Sea $A$ una sentencia rectificada en forma normal negada, la \textbf{forma normal de Skolem de A} (\textbf{SK(A)}) se define recursivamente como sigue:

Sea $A'$ cualquier subfórmula de $A$, 
\begin{itemize}
\item Si $A'$ es una fórmula atómica o su negación, \textbf{SK}$(A') = A'$.
\item Si $A'$ es de la forma $(B\star C)$ con $\star \in \{\land,\lor\}$, entonces $\textbf{SK}(A') = (\textbf{SK}(B)\star \textbf{SK}(C))$.
\item Si $A'$ es de la forma $\forall x.B$, entonces $\textbf{SK}(A') = \forall x.\textbf{SK}(B)$.
\item Si $A'$ es de la forma $\exists x.B$ y $\{x,y_1,\dots,y_m\}$ son las variables libres de $B$, entonces:
\begin{enumerate}
\item Si $m>0$, crear un \textbf{símbolo de función de Skolem}, $f_x$ de aridad $m$ y definir:

$$\textbf{SK}(A') = \textbf{SK}(B\{ x \leftarrow f(y_1,\dots,y_m)\})$$

\item Si $m=0$, crear una nueva \textbf{constante de Skolem} $c_x$ y

$$\textbf{SK}(A') = \textbf{SK}(B\{ x \leftarrow c_x\})$$

\end{enumerate}
\end{itemize}

\subsection{Reglas de resolucion de primer orden}
$$\frac{\{\blue{B_1,\dots,B_k},A_1,\dots,A_n\}\hspace*{5mm}\{\blue{\lnot D_1,\dots,\lnot D_k},A_1,\dots,A_n\}}{\red{\sigma(\{A_1,\dots A_m, C_1,\dots,C_n\})}}$$

donde $\sigma$ es el \textbf{unificador más general} (MGU) de $\{B_1,\dots,B_k,\lnot D_1,\dots,\lnot D_k\}$ y \\ $\sigma(\{A_1,\dots A_m, C_1,\dots,C_n\})$ es el \textbf{resolvente}.

\subsection{Regla de resolucion binaria y factorizacion}
$$\frac{\{\blue{B_1},A_1,\dots,A_n\}\hspace*{5mm}\{\blue{\lnot D_1},A_1,\dots,A_n\}}{\red{\sigma(\{A_1,\dots A_m, C_1,\dots,C_n\})}}$$

$$\frac{\{\blue{B_1,\dots,B_k},A_1,\dots,A_n\}}{\red{\sigma(\{B_1,A_1,\dots A_m\})}}$$

\section{Prolog predicados}
\paragraph{Predicados:} 
=, sort, msort, length, nth1, nth0, member, append, last, between, is\_list, list\_to\_set, is\_set, union, intersection, subset, subtract, select, delete, reverse,  atom,  number, numlist, sumlist, flatten, help

\paragraph{Operaciones extra-lógicas}: is, $\backslash =$,  $==$,  $=:=$, $=\backslash=$, $>$, $<$, $=<$, $>=$, abs, max, min, gcd,  var,  nonvar, ground, trace, notrace

\paragraph{Metapredicados:} bagof, setof, maplist, include, not, forall, assert, retract, listing
\end{document}