
\documentclass[10pt,a4paper]{article}
\usepackage[spanish]{babel}
\usepackage[utf8]{inputenc}

 
\usepackage[pdftex,
pdfauthor={Gianfranco Zamboni},
pdftitle={Paradigmas de Lenguajes de Programación},
pdfsubject={},
pdfkeywords={Resumen , Computacion, FCEyN, UBA, Paradigmas de Lenguajes de Programación, Imperativo, Funcional, Cálculo Lambda, Programación Orientada a Objetos, Objetos, Programación Lógica, Recursion, Tipado, Sintaxis, Subtipado, Semantica, Programacion orientada a objetos},
pdfproducer={Latex with hyperref},
pdfcreator={pdflatex},
pdfencoding = auto]{hyperref}

\usepackage{a4wide}
\usepackage{amsmath}
\usepackage{amssymb}
\usepackage{fancyhdr}
\usepackage{forest}
\usepackage{tikz}
\usepackage{minted}
\usepackage{multicol}
\usepackage{mathabx}
\usepackage{bussproofs}

\pagestyle{fancy}
\fancyhf{}
\renewcommand{\sectionmark}[1]{\markright{\thesection\ - #1}}
\fancyhead[LO]{Sección \rightmark }  
\fancyhead[RO]{\small{Paradigmas de Lenguajes de Programación}}
\fancyfoot[CO]{\thepage}
\renewcommand{\headrulewidth}{0.5pt}
\renewcommand{\footrulewidth}{0.5pt}
\setlength{\headsep}{1cm}
\setlength{\headheight}{13.07225pt}

\renewcommand{\baselinestretch}{1.2}  % line 

\setcounter{tocdepth}{2}% Allow only \chapter in ToC


\newenvironment{centrado}
    {
     \begin{center}
     \begin{minipage}{0.8\textwidth}
 }    
    {
     \end{minipage}
     \end{center}
    }

\newcommand{\rel}{\ensuremath{\mathcal{R}}}

\newcommand{\equalDef}{\overset{def}{=}}
\newcommand{\equalDot}{\overset{\cdot}{=}}

\newcommand{\lambdaAbs}[3]{\lambda #1: #2 . #3}
\newcommand{\lambdaAssign}[2]{#1~:=~#2}
\newcommand{\lambdaApp}[2]{#1~#2}
\newcommand{\lambdaIf}[3]{if~ #1~ then~ #2~ else~ #3}
\newcommand{\lambdaTrue}{true}
\newcommand{\lambdaFalse}{false}
\newcommand{\lambdaLet}[4]{let~#1:#2 = #3~in~#4}
\newcommand{\lambdaRef}[1]{ref~#1}
\newcommand{\lambdaVar}[1]{#1}
\newcommand{\lambdaValue}[1]{\color{red}#1\color{black}}
\newcommand{\lambdaFix}[1]{fix~#1}

\newcommand{\lambdaAbsI}[2]{\lambda #1. #2}
\newcommand{\lambdaLetI}[3]{let~#1 = #2~in~#3}



\newcommand{\blue}[1]{\color{blue}#1\color{black}}
\newcommand{\replaceBy}[3]{#1\{#2\leftarrow#3\}}

\newcommand{\judgeType}[3]{#1\triangleright #2 : #3}
\newcommand{\judgeTypeS}[3]{#1\mapsto #2 : #3}


\newcommand{\OOAtributo}[3]{#1 = \varsigma(#2)#3}
\newcommand{\OORedefinicion}[3]{#1 \leftleftharpoons \varsigma(#2)#3}
\newcommand{\OOReduccion}[2]{#1 \longrightarrow #2}
\newcommand{\OORep}[1]{\green{[[}#1\green{]]}}

\newenvironment{scprooftree}
{\leavevmode\hbox\bgroup}
{\DisplayProof\egroup}

\tikzset{
    every leaf node/.style={text=red, align=center},
    every tree node/.style={text=blue, align=center},
}

\forestset{tikzQtree/.style={for tree={if n children=0{
                node options=every leaf node/.try}{node options=every tree node/.try}, text centered}}}
                
                
\DeclareMathOperator{\Erase}{Erase}

\newcommand{\List}[1]{[~]_{#1}}
\newcommand{\lambdaListCase}[3]{\text{case } #1 \text{ of } \{[~]\leadsto#2~|~h::t\leadsto#3\}}

\newcommand{\WFunc}{\mathbb{W}}

\newcommand{\red}[1]{{\color{red}#1}}%\renewcommand{\appendixtocname}{Apéndices}
\newcommand{\green}[1]{{\color{green!60!black}#1}}%\renewcommand{\appendixpagename}{Apéndices}





\setcounter{section}{6}


\begin{document}
  \title{PLP - Práctica 7: Programación Lógica}

  \date{\today}

  \author{Zamboni, Gianfranco}

  \maketitle
  \setcounter{page}{1}


\section*{\ El motor de búsqueda de prolog}
\subsection{Ejercicio 1}
Base de conocimiento:
\begin{itemize}
    \item padre(juan, carlos).
    \item padre(juan, luis).
    \item padre(carlos, daniel).
    \item padre(carlos, diego).
    \item padre(luis, pablo).
    \item padre(luis, manuel).
    \item padre(luis, ramiro).
    \item abuelo(X,Y) :- padre(X,Z), padre(Z,Y).
\end{itemize}
Respuestas:
\begin{enumerate}
\item abuelo(X, manuel): padre(X,Z), padre(Z,manuel) \\
Juan
\item Definición:
hijo(X,Y) :- padre(Y,X). \\
hermano(X,Y) :- padre(Z,X), padre(Z,Y). \\
descendiente(X,Y) :- padre(Y,X). \\
descendiente(X,Y) :- padre(X,Z), descendiente(Z,Y).
\item \blue{arbol}
\item abuelo(Juan,X).
\item hermano(Pablo,X).
\item Regla: \\
ancestro(X, X). \\
ancestro(X, Y) :- ancestro(Z, Y), padre(X, Z).
\item el problema es que se genera una consulta que no termina, porque todo el tiempo el arbol se abre por ancestro().
\item Solucion: \\
ancestro(X, X). \\
ancestro(X, Y) :- padre(X, Z), ancestro(Z, Y).
\end{enumerate}
\subsection{Ejercicio 2}
\subsection{Ejercicio 3}
\section*{\ Estructuras, instanciación y reversibilidad}
\subsection{Ejercicio 4}
\subsection{Ejercicio 5}
\subsection{Ejercicio 6}
\subsection{Ejercicio 7}
\subsection{Ejercicio 8}
\subsection{Ejercicio 9}
\subsection{Ejercicio 10}
\subsection{Ejercicio 11}
\subsection{Ejercicio 12}
\section*{\ Generate and test}
\subsection{Ejercicio 13}
\subsection{Ejercicio 14}
\subsection{Ejercicio 15}
\section*{\ Negación por falla}
\subsection{Ejercicio 16}
\subsection{Ejercicio 17}
\subsection{Ejercicio 18}
\subsection{Ejercicio 19}

\subsection{Ejercicio 20}
\subsection{Ejercicio 21}
\subsection{Ejercicio 22}

\end{document}
